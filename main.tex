\documentclass[shuuron]{kuee} % ←ここが修士化の本丸

% ===== packages =====
\usepackage{booktabs}
\usepackage{amsmath, amssymb, amsfonts}
\usepackage{mathtools}
\usepackage{algorithm}
\usepackage{algorithmic}
\usepackage[dvipdfmx]{graphicx}
\usepackage{url}
\usepackage{multirow}

% 図の探索パス(必要なら)
\graphicspath{{figures/}}

\DeclareMathOperator*{\argmax}{arg\,max}

% ===== cover info =====
\title{不確実環境における\\適応的モデルベース強化学習法}
\etitle{Adaptive Model-Based Reinforcement Learning in Uncertain Environments}
\author{志賀 泰斗}
\eauthor{Taito Shiga}
\professor{石井 信 教授}
% course/department は cls 側で修士用が既定値に切り替わりますが、
% 表示を上書きしたいなら \course{...} \department{...} を書けます。:contentReference[oaicite:3]{index=3}
\date{令和8年1月提出}

\begin{document}
\maketitle

\begin{eabstract}
% English abstract here
モデルベース強化学習は,環境の遷移ダイナミクスを学習し,
その予測に基づいて行動を計画することで高いサンプル効率を実現する.
しかし,学習時に想定された物理パラメータから実行時の環境が乖離した場合,
世界モデルの予測精度が低下し,計画の信頼性が損なわれるという課題がある.
本研究では,物理パラメータが未知であり,エピソード内で一定または変動する環境において,
過去の相互作用履歴から環境特性を明示的に推定し,
その推定結果を世界モデルおよび計画過程に条件として与えることで,
環境変動下でも整合的な将来予測を実現する適応的制御手法を提案する.
まず,真の物理パラメータを与えた Oracle 条件での検証により,
物理パラメータの明示的な情報が計画の信頼性向上に寄与することを確認し,
提案手法のアプローチの妥当性を示した.
その上で,Pendulum-Swingup および Walker-Walk タスクを用いた実験により,
提案手法が高次元タスクにおいて既存手法を上回る性能を達成し,
動的環境においても環境変化に追従できることを実証した.
\end{eabstract}

\tableofcontents

% ===== chapters =====
\chapter{序論}

近年,ロボット技術は産業,物流,災害対応など,
さまざまな分野において実社会への展開が進んでいる。
これらの応用では,
事前にすべての状況を想定して制御則を設計することが困難であり,
環境や状況に応じて柔軟に振る舞いを変化させる能力が
ロボットに求められている。
このような背景から,
試行錯誤を通じて行動方策を獲得する学習的アプローチが
重要性を増している。

その代表的な枠組みとして,
環境との相互作用を通じて意思決定方策を学習する
強化学習(Reinforcement Learning; RL)が注目されてきた。
近年では,深層ニューラルネットワークを関数近似器として用いることで,
高次元かつ非線形な制御問題に対しても
高い性能を示すことが報告されている
\cite{hafner2020mastering,hafner2019dream,lee2020stochastic}。
RLは,事前に環境モデルを厳密に設計することなく,
複雑な制御問題に適用可能である点から,
実世界ロボットへの応用が期待されている。

一方で,実社会での運用を想定した場合,
実環境における試行錯誤には
安全性やコストの観点から大きな制約が存在する。
このため,
限られたデータから効率的に学習可能な手法が求められる。
この文脈において,
環境の将来挙動を内部モデルとして学習し,
その予測に基づいて行動を計画する
モデルベース強化学習
(Model-Based Reinforcement Learning; MBRL)
は,サンプル効率の観点から
実運用に適した手法として期待されてきた
\cite{NIPS2014_c7c9344b}。
近年では,
高次元観測を潜在空間に写像し,
その空間上で将来の行動系列を計画する手法が提案されており,
TD-MPC2はその代表的な手法の一つである。

しかし,実環境への適用を考えた場合,
既存の多くのMBRL手法は依然として重要な課題に直面している。
モデルベース強化学習は,
学習時に獲得された環境ダイナミクスに基づいて計画を行うため,
物理パラメータが変動した場合,
その影響が計画性能に直接的に現れ, 性能劣化を引き起こすことがある。

この問題が顕在化する代表的な例の一つが,
シミュレーション環境で学習した方策を
実環境へと転移する際に生じる
Sim2Real Gap である。
シミュレーションと実環境の間には,
物理パラメータや接触特性などに差異が存在するため,
転移後に次状態予測の誤差が増大し,
計画の破綻や制御性能の低下を招くことがある。

さらに,
学習と実行の環境が一致している場合であっても,
実環境内では時間の経過とともに
ダイナミクスが変化する可能性がある。
ロボットの摩耗や経年劣化,
個体差,部品交換などにより,
運用中に物理パラメータが変動し,
学習時とは異なるダイナミクスに直面することがある。
このような実環境内での変化もまた,
MBRL手法の性能劣化を引き起こす要因となる。

このように,
Sim2Real における転移時のギャップと,
実環境内で生じるダイナミクス変動は,
発生要因や時間的スケールは異なるものの,
いずれも物理パラメータの変動によって
学習時の世界モデルと実行時の環境ダイナミクスが乖離する点で共通している。
固定的な世界モデルに基づく計画は,
この乖離に対して脆弱であり,
結果として性能の劣化を招きやすい。

以上の背景から,
実運用上起こりうる物理パラメータの変動に対して,
性能劣化を抑制可能なMBRL手法の構築が重要である。
本研究では,
TD-MPC2を基盤とし,
過去の挙動履歴から環境に固有な物理特性を推定し,
その情報を世界モデルの予測に反映させることで,
環境ダイナミクスの変化に適応する
モデルベース強化学習の枠組みを提案する。
本手法では,
環境特性の同定と,
それに基づくダイナミクス予測とを
概念的に分離して扱うことで,
物理パラメータ変動下においても
計画性能を維持することを目指す。

本研究の主な貢献を以下に示す。

\begin{itemize}
  \item \textbf{物理パラメータ変動下における課題構造の整理}  
  
  物理パラメータの変動が,
  モデルベース強化学習の計画性能に与える影響を整理し,
  固定的な世界モデルに基づく手法の限界を明確にした。

  \item \textbf{環境特性同定に基づく適応的MBRL手法の提案}  
  
  TD-MPC2を基盤とし,
  環境に固有な物理特性を明示的に同定し,
  その情報を世界モデルに反映させる
  適応的なモデルベース強化学習手法を提案した。

  \item \textbf{物理パラメータ変動に対する有効性の検証} 
 
  Pendulum-SwingupおよびWalker-Walkなどの制御タスクを用いて,
  物理パラメータ変動下における性能を評価し,
  提案手法が非適応型ベースラインに対して
  有効であることを示した。
\end{itemize}

\chapter{研究背景}
\label{chap:background}

本章では,本研究の背景として,強化学習およびモデルベース強化学習(MBRL)の基本概念と,
実環境適用において顕在化する課題を整理する.
まず\ref{chap:rl}節で強化学習の基礎を概説する.
続いて\ref{sec:mbrl}節で,MBRLの枠組みと近年の進展を概観する.
その上で,\ref{sec:uncertainty_mbrl}節では,学習モデルに基づく計画という性質が,
環境変動下でどのような不確実性として現れるかを整理する.
さらに\ref{sec:existing_strategies}節で,不確実性に対する既存の戦略を概観し,
最後に\ref{sec:our_position}節で本研究の立場を明確にする.

\section{強化学習}
\label{chap:rl}

\subsection{基礎理論}
\label{chap:fund_rl}
強化学習(Reinforcement Learning:RL)は,エージェントが環境内での行動を通じてタスクを学習することを目的とした
一種の機械学習手法である\cite{sutton2018reinforcement}.このプロセスでは,エージェントは状態から行動を選択し,
各状態における状態遷移確率にしたがい次の状態へ遷移する.エージェントはそれぞれの遷移に対して報酬を受け取り,この報酬の最大化を目指す
最適化問題を解くことによって,どの行動系列が最も有益であるのかを学習する.
重要なのは,エージェントが特定の指令を受けることなく,試行錯誤を通じて環境と相互作用することによってのみ学習が進められることである.この過程
においてエージェントは即時に得られる報酬だけでなく,将来得られるであろう累積報酬予測を考慮して行動を選択しなければならない.強化学習
が扱う問題は多くの場合,マルコフ決定過程(Markov Decision Process: MDP)\cite{markov}としてモデル化された環境内で行動し,
エージェントはその環境からの十分な観測を通じて,タスクの遂行に適切な状態遷移を引き起こす行動をとる必要がある.
上で述べたMDPは,タプル$\mathcal{M}=(\mathcal{\mu_{0}}, \mathcal{S}, \mathcal{A}, \mathcal{P}, \mathcal{R}, \gamma)$で表現される.
ここで,$\mathcal{\mu_{0}}$は初期状態空間を,$\mathcal{S}$は状態空間を,$\mathcal{A}$は行動空間を表す.$\mathcal{P}$は確率関数の集合で,
$\mathcal{P}:\mathcal{S}\times\mathcal{A}\times\mathcal{S}\rightarrow\mathbb{R}$であり,$\mathcal{P}(s'|s,a)$は
状態$s$から行動$a$をとったときに状態$s'$へ遷移する確率を表す.また,$\mathcal{R}:\mathcal{S}\times\mathcal{A}\times\mathcal{S}\rightarrow\mathbb{R}$は報酬関数であり,
状態$s$から行動$a$をとったときに状態$s'$へ遷移するときに得られる報酬を表す.そして,$\gamma\in(0,1]$は割引率と呼ばれ,
将来得られる報酬の重要度を決定する.

エージェントは$\mathcal{M}$の中で,
状態$s$において行動$a$をとる確率が$\pi(a|s)$で与えられる確率的な方策$\pi:\mathcal{S}\rightarrow\mathcal{A}$に基づいて行動する.
あるMDP$\mathcal{M}$と方策$\pi$が与えられると状態価値関数$V_{\mathcal{M}}^{\pi}(s)$が以下の式(\ref{eq:V})で考えられる.
\begin{equation}
    V_{\mathcal{M}}^{\pi}(s) = \mathbb{E}\left[\sum_{t=0}^{T}\gamma^{t}r_{t}|\pi,s\right]
    \label{eq:V}
\end{equation}
ここで,$r_{t}$は時刻$t$における報酬,$T$はエージェントがタスクが完了するまで$\mathcal{M}$内で行動する回数,すなわちエピソードの長さを
表す.エピソードに終了条件を設定しないnon episodicなタスクの場合,$T=\infty$となる.
期待値$\mathbb{E}$は$s_0\sim\mu_0$,$a_i\sim\pi(\cdot|s_i)$,$s_{i+1}\sim\mathcal{T}(\cdot|s_i,a_i)$で取る.
状態価値関数は状態$s$からこの先方策$\pi$に基づいて行動した場合に得られるであろう割引累積報酬の期待値を表し,その状態にいる価値を推定する.
また,状態$s$から行動$a$を取ったときに期待される割引累積報酬を推定する行動価値関数$Q_{\mathcal{M}}^{\pi}(s,a)$が式(\ref{eq:Q})で導出できる.
\begin{equation}
    Q_{\mathcal{M}}^{\pi}(s,a) = \mathbb{E}_{s'\sim\mathcal{T}(\cdot|s,a)}[\mathcal{R}(s,a,s')+{\gamma}V_{\mathcal{M}}^{\pi}(s')]
    \label{eq:Q}
\end{equation}

以上からRLの基本となるゴールは,$Q_{\mathcal{M}}^{*}(s,a)=\sup\limits_{\pi}Q_{\mathcal{M}}^\pi(s,a)$
なる最適な状態および行動価値関数から最適な方策$\forall s \in \mathcal{S},\pi_{\mathcal{M}}^{*}(s)=\argmax\limits_{a \in \mathcal{A}}Q_{\mathcal{M}}^{*}(s,a)$
を得ることである.よって方策更新により最大化される目的関数$\mathcal{J}(\pi)$は
\begin{equation}
    \mathcal{J}(\pi) = \mathbb{E}_{(s,a)\sim\mu^{\pi}(s,a)}[\mathcal{R}(s,a,s')]
    \label{eq:pi}
\end{equation}
として表される.ここで$\mu^{\pi}(s,a)$は方策$\pi$に基づく定常状態行動確率分布であり,
この方策の下でエピソードにわたって行動を続けたときの,状態と行動のペアの長期的な分布を表す\cite{bellemare2017distributional}.
また,RLの問題は予測と方策改善の二つのフェーズに分けられる.予測の段階では現在の方策の質が評価され,方策改善フェーズでは
予測で評価された方策を改善するよう方策が調節される.RLアルゴリズムはこれらの二つのステップを反復して方策の最適化を行う.

\subsection{TD学習}
\label{sec:tderror}
Temporal Difference(TD)学習は,TD誤差と呼ばれる,1ステップ先の期待報酬の推定値と現在の期待報酬の推定値との差分
を用いて価値関数を更新する手法である\cite{sutton1988learning}.TD学習はモデルフリーな予測を行う
アルゴリズムであり,エージェントはMDPの情報が与えられない.よってエージェントはMDPのサンプルから
状態価値を推定する.状態価値関数の更新則は以下の式\ref{eq:td}の通りとなる.
\begin{equation}
    \label{eq:td}
    V_{\mathcal{M}}^{\pi}(s) \leftarrow V_{\mathcal{M}}^{\pi}(s) + \alpha(r_{t+1} + \gamma V_{\mathcal{M}}^{\pi}(s_{t+1}) - V_{\mathcal{M}}^{\pi}(s_{t}))
\end{equation}
ここで$\alpha$は学習率を表し,$r_{t+1}$は現在のステップ$t$で行動した結果,次のステップ$t+1$に遷移する際に得られた報酬である.
$r_{t+1}+\gamma V_{\mathcal{M}}^\pi$はTD目標と呼ばれ,今回のサンプリングで得られた値を,
$r_{t+1}+\gamma V_{\mathcal{M}}^\pi-V_{\mathcal{M}}^\pi$はTD誤差で,今回のサンプリングの値と前回までに得た値の差を表している.
TD学習の利点は,学習に状態価値関数の推定値を用いているため,エージェントがエピソードを完了していなくても
その状態から先の報酬を予測(ブートストラップ)し,学習を1ステップからでも行うことが可能である.

以上のように,価値推定と方策改善の枠組みは強化学習の基礎を成す.
しかし,実環境での試行錯誤には安全性やコストの制約が大きく,
限られたデータから効率的に学習する枠組みが必要となる.
この観点から,環境モデルを明示的に学習し計画に利用するモデルベース強化学習が検討されてきた.

% モデルベース強化学習の進展
% TODO: dreamerとかも引用して、tdmpc2に偏った議論を避けるようにする
\section{モデルベース強化学習の進展}
\label{sec:mbrl}

モデルベース強化学習(Model-Based Reinforcement Learning; MBRL)は,
環境の遷移ダイナミクス $p(s_{t+1} \mid s_t, a_t)$ を近似する学習モデル
$\hat{f}_{\theta}$ を構築し,これを利用して行動を決定する手法である\cite{kaelbling1996reinforcement}.
MBRLの最大の特徴は,学習済みモデル上で将来の状態遷移を予測し,
その予測に基づいた計画(Planning)が可能である点にある.
一般にMBRLは,経験を直接方策に反映させるモデルフリー手法と比較して
サンプル複雑性の面で優れており \cite{nagabandi2018neural,NIPS2014_c7c9344b,deisenroth2013gaussian},
実機実験のコストや安全性が懸念されるロボティクス分野において
極めて重要な役割を果たす.

初期のMBRLにおいては,
ガウス過程(Gaussian Processes)\cite{deisenroth2011pilco} や
時間変化線形モデル \cite{levine2014learning}
といった比較的単純な関数近似器を用いたダイナミクス学習が主流であった.
特に PILCO \cite{deisenroth2011pilco} は,
確率的なダイナミクスモデルとモデルの不確実性を
長期的な計画に組み込むことで,
優れたサンプル効率を達成した.
しかし,これらの古典的な近似手法は,
高次元の状態空間や摩擦接触を含む複雑な非線形ダイナミクスを
正確に捉えることが困難であるという課題を有していた
\cite{nagabandi2018neural}.

これに対し,近年では深層ニューラルネットワーク
(Deep Neural Networks; DNN)を
高容量な関数近似器として導入することで,
高次元かつ複雑なダイナミクスを持つ
ロボット制御タスクへの適用が可能となっている.

DNNを用いたモデル学習においては,
高次元の観測値を直接予測する代わりに,
制御に必要な情報を凝縮した低次元の潜在空間
(Latent Space)上でダイナミクスを記述する手法が
主流となっている.
これにより,高次元観測に伴う計算コストの抑制と,
長期的なホライゾンにおける予測精度の維持を
両立している.

エージェントは各時刻 $t$ において,
学習されたモデルを用いて以下の最適化問題を解くことで
行動系列を決定する.
\begin{equation}
(a_t, \dots, a_{t+H-1})
=
\argmax_{a_t, \dots, a_{t+H-1}}
\sum_{t'=t}^{t+H-1}
\gamma^{t'-t}
r(s_{t'}, a_{t'})
\label{eq:mbrl_planning}
\end{equation}
ここで $H$ は予測ホライゾンを表す \cite{nagabandi2018neural}.

この枠組みの代表例である TD-MPC2 \cite{hansen2023td} は,
潜在空間上での状態遷移,報酬予測,
および価値関数の学習を統合し,
サンプリングベースの最適化手法を用いることで,
複雑な連続値制御タスクにおいて高い性能を示している.
TD-MPC2はマルチタスク学習においても優れた能力を発揮する一方で,
内部に保持する世界モデルは
学習時の環境パラメータに固執した
静的な表現になりやすく,
運用中に生じる物理的な摂動への適応には
課題を残している.

% モデルベース強化学習における環境の不確実性

\section{モデルベース強化学習における環境不確実性}

MBRLは,\ref{sec:mbrl}で述べたように, 学習されたダイナミクスモデルに基づいて
将来の状態遷移を予測し,その予測結果を用いて計画を行う枠組みである.
このため,モデルの予測精度は制御性能に直接的かつ決定的な影響を及ぼす.
特に,複数ステップ先の状態遷移を考慮する計画型手法においては,
わずかな予測誤差であっても,その影響が時間とともに累積し,
行動選択の妥当性を大きく損なう可能性がある.

しかし,実環境への適用を考えた場合,
学習時に想定された環境と完全に同一の遷移ダイナミクスが
運用中も維持されるとは限らない.
ロボットの個体差,部品の摩耗や経年劣化,
あるいは接触条件や外乱といった環境側の要因により,
実際の遷移ダイナミクスは学習時とは異なるものとなり得る.

このようなダイナミクスのミスマッチが生じた場合,
学習済みモデルに基づく状態予測は徐々に現実から乖離していく.
とりわけ,モデル予測制御(Model Predictive Control; MPC)に代表される
計画型アプローチでは,
予測誤差が行動計画全体に波及し,
意図しない状態遷移やタスク失敗を引き起こす要因となる.
この問題は,接触を含む複雑なタスクや,
長い予測ホライゾンを必要とするタスクにおいて
より顕著に現れることが知られている\cite{nagabandi2018learning,lee2020context}.

近年の高性能なMBRL手法においても,
この不確実性の問題は本質的には未解決である.
多くの手法では,学習過程を通じて獲得された
単一の環境ダイナミクスを内部モデルとして保持し,
それを固定的に用いた計画が行われる.
その結果,運用中に生じる物理パラメータの変動や
環境条件の変化に対して,
モデルを動的に修正・適応する機構を十分に備えていない場合が多い.

このように,現在のモデルベース強化学習手法は,
環境が不変であるという暗黙の前提の下で設計されており,
パラメータが継続的に変動し得る実世界環境においては,
適応能力の不足が実用上の大きな課題となっている.

% 不確実性に対する既存の戦略

\section{環境不確実性に対する既存の戦略}
\label{sec:existing_strategies}

前節で述べたように,モデルベース強化学習(MBRL)は,
学習されたダイナミクスモデルに基づいて計画を行うため,
環境ダイナミクスの変動に起因する予測誤差が
制御性能に直接的な影響を及ぼす.
このようなダイナミクスのミスマッチに対処するため,
これまでの研究では大きく分けて
\emph{頑健化(Robustness)}と
\emph{適応化(Adaptation)}
という二つの戦略が検討されてきた.

頑健化の代表的な手法として,
Domain Randomization(DR)が広く用いられている\cite{tobin2017domain}.
DR では,学習段階においてリンクの質量や摩擦係数などの
物理パラメータを一定の範囲内でランダムに変化させることで,
単一の方策が多様な環境条件下でも
平均的に良好な性能を発揮することを目指す.
このアプローチは実装が比較的容易であり,
Sim2Realの文脈においても一定の成功を収めてきた.

一方で,DR はあらゆる環境変動に対して
単一の方策で対応することを前提とするため,
特定の環境条件における最適性を犠牲にした
保守的な制御になりやすいという課題を有する.
また,想定された分布の外側にある変動や,
時間的に変化するパラメータに対しては,
十分な性能を維持できない場合も少なくない\cite{ding2021not}.

これに対し,実行時に環境の変化を検知し,
モデルや方策を動的に調整する
適応化のアプローチが提案されてきた.
この系譜には,過去の遷移履歴を用いて
モデルパラメータを更新するメタ学習的手法\cite{nagabandi2018learning}や,
リカレント構造の内部状態を更新することで
環境変化に追従する手法が含まれる\cite{duan2016rl}.

近年では,環境の特性を
「コンテキスト」と呼ばれる潜在変数として抽出し,
それを条件付け情報として
ダイナミクスモデルや方策に入力する
コンテキスト学習の枠組みが注目を集めている\cite{lee2020context}.
このような手法では,
未知の環境要因や複合的な変動を
潜在表現に吸収することで,
幅広い環境条件への適応が可能となる.
その結果,頑健化手法と比較して,
特定の環境に特化した鋭い制御を実現できる点が
大きな利点とされている.

しかし,多くの既存手法において,
抽出されるコンテキストは
物理的な意味を持たない潜在表現であり,
何が原因でダイナミクスが変化したのかを
明確に解釈することは困難である.
そのため,性能劣化が生じた際の
失敗要因の特定や,
実運用における信頼性評価・デバッグを
体系的に行うことが難しいという問題が残されている.

% \section{本研究の立場}
% \label{sec:our_position}

% 潜在コンテキストによる暗黙的適応は,
% 未知要因を広く包含できる可能性がある一方で,
% 推定された表現の解釈や
% 失敗要因の特定が困難になりやすい.
% 本研究は,実運用における
% 計画の信頼性と診断可能性を重視し,
% 環境変動を
% \emph{明示的に同定すべき対象}
% として扱う立場を取る.

% 具体的には,環境ダイナミクスの変化を
% 潜在表現に吸収するのではなく,
% 物理パラメータとして切り出して推定し,
% その推定結果を世界モデルへ直接反映する枠組みを採用する.
% この設計により,
% 計画精度の向上だけでなく,
% どの物理的要因が制御性能に影響を与えているのかを
% 定量的に分析することが可能となる.

\section{本研究の立場}
\label{sec:our_position}

本研究では,環境変動に対する適応を,
単に性能を維持するための経験的工夫としてではなく,
モデルベース強化学習における計画性能劣化の原因を
構造的に切り分ける問題として捉える.
特に,物理パラメータの不一致が世界モデルの予測誤差を通じて
計画の破綻として顕在化する点に着目する.

この観点から,
環境ダイナミクスの変化を
潜在表現に吸収するのではなく,
\emph{明示的に同定すべき物理条件}として扱う立場を取る.
具体的には,真の物理パラメータが既知であるとする
Oracle条件下において,
計画性能が摂動に対して安定することをまず確認し,
計画が成立するための構造的条件を明らかにする.

一方で,実環境では真の物理パラメータに直接アクセスできないため,
その近似として推定機構を導入し,
Oracle条件に近い世界モデルを構成することで
実用的な適応を実現する.
このように本研究は,
物理パラメータ推定そのものを目的とするのではなく,
計画が安定する条件を明示的に切り出し,
その条件を近似する枠組みとして推定を位置づける.


% 言いたいこととしては、ちゃんと推定できているからうまくいっている、
% 既存の研究では暗示的推定機構を加えることで、その潜在空間の分析などからこれを入れるとうまくいくだろう
% って言っているだけで、そこは定量評価を行う観点では限界が存在する。
% それに対して、oracleを導入していることで、明示的な物理条件の下では摂動に対して強いパフォーマンスが
% 得られる、そのうえでoracleにはアクセスができないから推定機構を加えている、という立場を強調したい。
\chapter{前提知識}
\label{chap:preliminary}

本章では,本研究の基盤となる潜在空間における世界モデルと,
それを用いた制御アルゴリズムである TD-MPC2 の数理的定義について述べる.
また,環境の不確実性に対してオンラインで適応するために不可欠な,
相互作用履歴に基づく系列推論の枠組みについて整理する.


\section{潜在空間モデル予測制御:TD-MPC2}
\label{sec:tdmpc2}

TD-MPC2 は,
学習された世界モデル上での計画(Planning)と,TD学習による価値推定を統合した,
MBRLの手法である\cite{hansen2023td}.
本節では,
TD-MPC2 の背景にある世界モデルの概念とその変遷を整理した後,
モデル予測制御の枠組み,
および TD-MPC2 の具体的なアーキテクチャについて順に説明する.


\subsection{世界モデルの概念と変遷}
\label{subsec:world_model}

世界モデル(World Model)とは,
エージェントが環境の振る舞いを模倣した「内部シミュレータ」を学習し,
その内部モデルを用いて将来の状態や報酬を予測する枠組みを指す\cite{ha2018world}.
この概念は,
人間が行動を起こす前に「もしこの行動を取ったら何が起こるか」を頭の中で想像する,
いわゆるメンタルモデルの考え方に着想を得ている.

\paragraph{高次元観測の圧縮}

ロボットが取得する画像やセンサデータなどの観測 \(\mathbf{o}_t\) は高次元であり,
そのままの空間でダイナミクスを学習し,将来を予測することは,
計算コストおよびデータ効率の観点から困難である.
そこで世界モデルでは,
観測 \(\mathbf{o}_t\) を低次元のベクトルである潜在表現
\(\mathbf{z}_t\) に写像し,
以降の予測や計画をこの潜在空間上で行う.
この圧縮により,
環境の本質的な状態のみを抽出し,
効率的な将来予測が可能となる.

\paragraph{再構成型世界モデル}

初期の世界モデル研究では,
潜在表現 \(\mathbf{z}_t\) が観測の情報を十分に保持していることを保証するため,
Variational Autoencoder (VAE) を用いた再構成型の学習が広く用いられていた\cite{ha2018world}.
VAE では,
観測 \(\mathbf{o}_t\) から潜在表現 \(\mathbf{z}_t\) を推定するエンコーダと,
\(\mathbf{z}_t\) から元の観測を復元するデコーダを同時に学習する.
このとき,
潜在表現は「元の観測を再現できること」を目的として最適化される.

この設計により,
潜在表現は観測空間の情報を網羅的に保持するが,
背景の模様や照明条件など,
制御に直接関係しない視覚情報も含めて学習することになる.
その結果,
モデルの表現能力や学習資源が,
制御に不要な要素にも割かれる可能性がある.

\paragraph{制御中心型(陰的)世界モデル}

これに対し,
近年の制御指向の世界モデルでは,
観測の再構成を明示的な目的としない設計が採用されている.
本研究のベースラインである TD-MPC2 も,
デコーダを持たない陰的世界モデル(Implicit World Model)を採用している.
このアプローチでは,
将来の報酬や価値を正確に予測するために必要な情報のみを潜在空間に保持し,
観測の忠実な復元は行わない.

具体的には,
観測 \(\mathbf{o}_t\) はエンコーダによって潜在状態 \(\mathbf{s}_t\) に変換され,
潜在状態と行動 \(\mathbf{a}_t\) に基づいて,
次の潜在状態 \(\mathbf{s}_{t+1}\) が予測される.
この潜在遷移を繰り返すことで,
将来の軌道を潜在空間上でロールアウトし,
それに対応する報酬および価値を評価することが可能となる.

\subsection{モデル予測制御 (Model Predictive Control, MPC)}
\label{subsec:mpc}

モデル予測制御(Model Predictive Control, MPC)は,
学習した内部モデルを用いて,
現在から \(H\) ステップ先までの行動系列を計画する制御手法である\cite{kouvaritakis2016model}.
時刻 \(t\) において,
MPC は次の期待累積報酬を最大化する行動系列
\(\mathbf{a}^*_{t:t+H-1}\) を求める.

\begin{align}
\mathbf{a}^*_{t:t+H-1}
=
\arg\max_{\mathbf{a}_{t:t+H-1}}
\mathbb{E}
\left[
\sum_{k=0}^{H-1} \gamma^k r_{t+k}
+
\gamma^H Q(\mathbf{s}_{t+H}, \mathbf{a}_{t+H})
\right].
\end{align}

ここで,
TD-MPC2 の特徴は,
有限の予測ホライゾン \(H\) の終端において,
学習済みの価値関数 \(Q\) を用いて,
それ以降の長期的なリターンを近似的に評価する点にある.
このブートストラップにより,
短い予測ホライゾンであっても,
長期的な影響を考慮した計画が可能となる.


\subsection{TD-MPC2 の具体的構成}
\label{subsec:tdmpc2_architecture}

TD-MPC2 は,
以下の 5 つのモジュールから構成される.

\begin{itemize}
\item \textbf{Encoder}:
観測 \(\mathbf{o}_t\) を潜在状態 \(\mathbf{s}_t\) に写像する.
\item \textbf{Latent Dynamics}:
潜在状態 \(\mathbf{s}_t\) と行動 \(\mathbf{a}_t\) から,
次の潜在状態 \(\mathbf{s}_{t+1}\) を予測する.
\item \textbf{Reward Model}:
潜在状態において得られる即時報酬 \(\hat{r}_t\) を予測する.
\item \textbf{Value Model}:
将来得られる累積報酬の期待値 \(\hat{q}_t\) を予測する.
\item \textbf{Policy Prior}:
MPC における探索を効率化するため,
有望な行動分布を事前に与える.
\end{itemize}

これらのモデルは,
多段階の潜在遷移予測に基づく損失関数を最小化することで,
同時に学習される.
さらに,
潜在表現のスケールを安定させる \textbf{SimNorm} や,
報酬および価値予測を分類問題として扱う
\textbf{離散回帰(Discrete Regression)} といった工夫により,
学習の安定性と実用性が向上している.


\section{相互作用履歴からの系列推論}
\label{sec:sequential_inference}

標準的な強化学習では,
現在の観測のみを用いて行動を決定することが多い.
しかし,
物理パラメータが未知であるような不確実環境では,
単一時刻の観測だけでは環境の状態を十分に同定できない.
このような状況に対処するため,
過去の相互作用履歴を考慮した系列推論が重要となる.


\subsection{不確実環境と部分観測}
\label{subsec:context_importance}

質量や摩擦係数といった物理パラメータは,
直接観測することができない一方で,
行動の結果として生じる挙動の差として時間的に現れる.
このように,
真の状態が完全には観測できない環境は,
部分観測マルコフ決定過程(Partially Observable Markov Decision Process, POMDP)
として定式化される.
POMDP においては,
過去の観測と行動の履歴を考慮した推論が不可欠となる.


\subsection{相互作用履歴の活用と GRU}
\label{subsec:gru}

過去 \(K\) ステップにわたる観測と行動の系列を,
\(\mathcal{H}_t = \{(\mathbf{o}_{t-K}, \mathbf{a}_{t-K}), \ldots, (\mathbf{o}_{t-1}, \mathbf{a}_{t-1})\}\)
と定義する.
この履歴情報から現在の環境特性を抽出するため,
本研究では再帰型ニューラルネットワーク(RNN)の一種である
Gated Recurrent Unit (GRU) を採用する.
GRU は,
内部状態をゲート機構によって制御することで,
長期的な依存関係を効率的に保持できる特徴を持つ.
これにより,
過去の相互作用に内在する挙動の違いを蓄積し,
未知の物理パラメータに関する情報を表現することが可能となる.

\chapter{提案手法}
\label{chap:proposal}

\section{不確実環境における制御問題の定式化}
\label{sec:problem_reformulation}

本節では,
第\ref{chap:preliminary}章で整理した TD-MPC2 の前提を踏まえ,
不確実な物理環境において生じる制御上の課題を明確化する.
その上で,
本研究が採用する基本的な設計方針について述べる.


\subsection{想定する環境不確実性と課題設定}
\label{subsec:uncertainty_setting}

本研究では,
ロボットの質量,関節特性といった
物理パラメータが未知な環境を環境不確実性として定義する.

このような物理パラメータは,
単一時刻の観測から直接推定することは困難であり,
エージェントの行動に対する環境の応答として,
時間的に蓄積される挙動の違いの中に現れる.
したがって,
現在の観測のみに基づく意思決定では,
環境の特性を十分に捉えることができない.

一方で,
エージェントが過去の観測および行動の履歴を考慮することで,
環境固有の物理特性に関する情報を間接的に取得できる可能性がある.
この点は,
部分観測環境における制御問題として捉えることができる.

\paragraph{物理パラメータの時間変化に関する想定}

本研究では,
物理パラメータの時間変化のパターンとして,
以下の二つの状況を想定する.

\begin{enumerate}
\item \textbf{エピソード間の変動}:
物理パラメータはエピソード開始時に決定され,
単一エピソード内では一定に保たれるが,
エピソードごとに異なる値をとる場合.
これは,
異なる個体や環境条件下での制御を想定したものであり,
学習したモデルが, 未知だが固定された物理パラメータ環境に対してどの程度汎化可能かを評価する設定として用いる.

\item \textbf{エピソード内の変動}:
物理パラメータがエピソード実行中に時間的に変化する場合.
これは,
バッテリー消耗によるアクチュエータ出力の低下や,
摩耗による機構特性の変化など,
実環境において生じうる動的な変動を想定したものである.
このような状況では,
推定器がオンラインで変化を追従する必要がある.
\end{enumerate}

提案手法の設計は,
いずれのパターンにも対応可能な構成を採用しており,
第\ref{chap:experiment}章の実験では両方の状況における性能を検証する.


\subsection{既存 TD-MPC2 における暗黙的仮定とその限界}
\label{subsec:tdmpc2_limitation}

第\ref{chap:preliminary}章で述べたように,
TD-MPC2 は,
学習された世界モデル上での将来予測と,
モデル予測制御に基づく計画を統合した強力な制御手法である.
しかし,
TD-MPC2 では,
世界モデルが固定された環境ダイナミクスを表現していることが暗黙的に仮定されている.

この仮定の下では,
学習時に想定していない物理パラメータの変動が生じた場合,
世界モデルによる将来予測と実際の環境挙動との間に乖離が生じる.
この乖離は,
モデル予測制御におけるロールアウトを通じて累積し,
計画の信頼性を低下させる要因となる.

\subsection{本研究の基本方針:不確実性の明示的同定}
\label{subsec:design_policy}

以上の背景を踏まえ,
本研究では,
環境の不確実性を単に吸収すべきノイズとして扱うのではなく,
\textbf{明示的に同定すべき対象}として捉える立場を取る.
すなわち,
環境の物理パラメータを推定し,
その推定結果を制御および計画に反映させることで,
将来予測の整合性を向上させることを目的とする.

本研究の設計目標は,
単に平均的な性能を向上させることではなく,
不確実環境下における
(i) 計画の信頼性,
(ii) 推定結果の解釈性
を同時に向上させることである.
この方針に基づき,
次節以降では,
明示的な物理パラメータ推定器と,
それを用いた適応的なモデル予測制御の構成について述べる.


\section{提案手法のシステム構成}
\label{sec:system_overview}

本節では,
本研究で提案する明示的物理パラメータ同定に基づく制御手法の
システム全体構成について述べる.
提案手法は,
既存の TD-MPC2 を制御器の中核として保持しつつ,
相互作用履歴から環境の物理パラメータを推定するモジュールを追加し,
その推定結果を計画過程に条件として与える構成を採用する.


\subsection{全体アーキテクチャ}
\label{subsec:overall_architecture}

提案手法の全体アーキテクチャを図\ref{fig:architecture}に示す.
本手法は,
大きく分けて
(i) 物理パラメータ推定モジュールと,
(ii) 条件付きモデル予測制御モジュール
の二つから構成される.

\begin{figure}[tb]
\centering
\includegraphics[width=0.9\linewidth]{figures/architecture.png}
\caption[提案手法の全体アーキテクチャ]{%
\centering\textbf{提案手法の全体アーキテクチャ}\\[1ex]
\raggedright\small
環境から得られる観測 $\mathbf{o}_t$ は,
まず明示的システム同定モジュールに入力され,
過去 $K$ ステップ分の観測–行動履歴
$(\mathbf{o}_{t-K:t}, \mathbf{a}_{t-K:t-1})$
を用いて GRU エンコーダにより
物理パラメータの潜在表現
$\hat{\mathbf{c}}_{\mathrm{phys}}$ が推定される.
同時に,現在の観測 $\mathbf{o}_t$ は
TD-MPC2 コントローラに入力され,
エンコーダ $f_{\mathrm{enc}}$ により潜在状態
$\mathbf{s}_t$ に写像される.
推定された物理パラメータ
$\hat{\mathbf{c}}_{\mathrm{phys}}$ は,
潜在ダイナミクスモデル,
報酬・価値モデル,
および方策事前分布に条件として与えられ,
物理パラメータ推定結果を反映した潜在ロールアウトが実行される.
MPPI プランナは,
条件付き世界モデルに基づいて行動系列を評価・最適化し,
得られた行動 $\mathbf{a}_t$ を環境に適用する.
}
\label{fig:architecture}
\end{figure}

制御器の中核には,
第\ref{chap:preliminary}章で述べた TD-MPC2 をそのまま用いる.
すなわち,
観測 \(\mathbf{o}_t\) はエンコーダによって潜在状態 \(\mathbf{s}_t\) に写像され,
潜在空間上の世界モデルを用いたロールアウトと
モデル予測制御により行動が決定される.

提案手法では,
制御とは独立に,
過去の観測および行動の履歴を入力として
環境の物理パラメータを推定する推定モジュールを追加する.
この推定結果は,
制御器内部の潜在状態とは別の変数として保持され,
世界モデルおよび計画過程に条件として与えられる.詳細は\ref{subsec:conditional_dynamics}節で述べる.

この構成により,
TD-MPC2 が本来持つ
高い計画性能および学習安定性を損なうことなく,
環境の物理的差異を明示的に反映した制御が可能となる.


\subsection{推定モジュールと制御モジュールの結合設計}
\label{subsec:module_coupling}

提案手法では,
物理パラメータ推定モジュールと制御モジュールを独立させた疎結合な構成を採用した.
推定された物理パラメータ \(\hat{\mathbf{c}}_{\mathrm{phys}}\) は,
制御器内部の潜在状態表現には統合せず,
各予測モジュールに対する「条件付け変数(conditioning variable)」として扱われる.
具体的には,
エンコーダを除く世界モデルの構成要素(Dynamics, Reward, Value)および行動事前分布(Policy Prior)の入力として \(\hat{\mathbf{c}}_{\mathrm{phys}}\) を明示的に注入する設計としている.

この設計の核心は,
推定プロセスを制御器の外部に配置することで,
物理パラメータ推定モジュールの学習が制御器側の表現獲得や価値学習を不安定化させることを防ぐ点にある.
学習時には推定値 \(\hat{\mathbf{c}}_{\mathrm{phys}}\) を介した勾配の逆伝播を遮断(detach)する構造をとることで,
既存の TD-MPC2 が備える優れた学習特性を損なうことなく,
環境特性に応じた適応能力のみを付加することが可能となる.
これにより,
エージェントは同一の潜在状態および行動に対しても,
同定された物理パラメータに応じて将来予測や計画結果を動的に変化させることができる.

このように推定と制御の役割を概念的に分離した構成は,
(i) 物理パラメータ同定の明示的な解釈性,
(ii) 不確実環境下における適応プロセスの安定性,
(iii) システム全体のモジュール性と実装の単純性
を同時に担保するものである.


\section{明示的物理パラメータ推定器}
\label{sec:explicit_sysid}

本節では,
相互作用履歴から環境の物理パラメータを推定するために導入した,
明示的物理パラメータ推定器について述べる.


\subsection{相互作用履歴に基づく推定問題の定式化}
\label{subsec:sysid_formulation}

時刻 \(t\) における過去 \(K\) ステップ分の観測および行動の履歴を,
次のように定義する.

\begin{align}
\mathcal{H}_t
=
\{(\mathbf{o}_{t-K}, \mathbf{a}_{t-K}), \ldots, (\mathbf{o}_{t-1}, \mathbf{a}_{t-1})\}.
\end{align}

本研究では,
この履歴情報 \(\mathcal{H}_t\) に基づいて,
環境に固有な物理パラメータ
\(\mathbf{c}_{\mathrm{phys}} \in \mathbb{R}^d\)
を推定する問題を考える.
ここで,
\(\mathbf{c}_{\mathrm{phys}}\) は,
質量や関節特性など,
制御性能に大きな影響を与えるが,
単一時刻の観測からは直接得られない量を表す.

そして推定問題は次の写像として表現できる.

\begin{align}
\hat{\mathbf{c}}_{\mathrm{phys},t}
=
g_{\theta}(\mathcal{H}_t),
\end{align}

ここで,
\(g_{\theta}(\cdot)\) は
履歴から物理パラメータを出力する推定器を表す.


\subsection{GRU による履歴情報のエンコーディング}
\label{subsec:gru_encoding}

相互作用履歴 \(\mathcal{H}_t\) は時系列データであり,
物理パラメータに関する情報は,
行動に対する環境応答の違いとして時間的に現れる.
このような時系列依存性を捉えるため,
本研究では再帰型ニューラルネットワーク(RNN)の一種である
Gated Recurrent Unit (GRU) を用いて履歴をエンコードする.

GRU は,
入力系列を逐次的に処理し,
隠れ状態 \(\mathbf{h}_t\) に過去の情報を集約する.
本研究では,
各時刻の入力を
観測と行動の組 \((\mathbf{o}_{t}, \mathbf{a}_{t})\)
として GRU に与え,
最終的な隠れ状態を
履歴全体の要約表現として用いる.

GRU を採用した理由は,
時系列情報を扱う RNN 系モデルの中でも構造が簡潔であり,
比較的少ないパラメータ数で安定した学習が可能であるためである.



\subsection{物理パラメータ回帰モジュール}
\label{subsec:parameter_regression}

GRU により得られた隠れ状態 \(\mathbf{h}_t\) を入力として,
物理パラメータを回帰するモジュールを構成する.
具体的には,
\(\mathbf{h}_t\) を全結合層に入力し,
推定値 \(\hat{\mathbf{c}}_{\mathrm{phys},t}\) を出力する.

本研究では,
実験設定の単純化のため,
推定対象の物理パラメータを一次元に限定する.
提案手法の構成は推定次元数に依存せず,
多次元の物理パラメータ推定にも同様に適用可能である.



\section{推定に基づく適応的計画}
\label{sec:adaptive_planning}

本節では,
前節で推定した物理パラメータを用いて,
モデル予測制御をどのように適応させるかについて述べる.
提案手法では,
推定結果を世界モデルおよび計画過程に条件として与えることで,
環境変動に応じた将来予測と行動計画を実現する.


% \subsection{物理パラメータの正規化と注入方法}
% \label{subsec:parameter_injection}

% 推定された物理パラメータ
% \(\hat{\mathbf{c}}_{\mathrm{phys},t}\) は,
% 世界モデルおよび計画アルゴリズムに入力される前に,
% 正規化処理を施す.
% これは,
% 学習時に用いたパラメータ分布と
% 推定値のスケールを整合させるためである.

% このような条件付けは,
% 物理パラメータが状態とは異なる時間スケールで変化するという仮定と整合的であり,
% 推定誤差が潜在状態推定に直接影響することを防ぐ効果も持つ.


\subsection{条件付き世界モデルによる将来予測}
\label{subsec:conditional_dynamics}

物理パラメータを条件として導入した場合,
潜在空間における状態遷移は次のように表される.

\begin{align}
\mathbf{s}_{t+1}
=
f(\mathbf{s}_t, \mathbf{a}_t, \hat{\mathbf{c}}_{\mathrm{phys},t}).
\end{align}

この条件付き遷移モデルにより,
同一の潜在状態および行動であっても,
物理パラメータの違いに応じて
異なる将来状態が予測される.
これにより,
世界モデルは
環境ごとのダイナミクス差異を明示的に表現することが可能となる.

条件付き世界モデルを用いたロールアウトでは,
推定された物理パラメータを固定したまま,
複数ステップ先までの潜在状態遷移を予測する.
この予測結果に基づき,
報酬および価値が評価され,
モデル予測制御における行動系列の最適化が行われる.


\subsection{推定パラメータ条件下での軌道計画}
\label{subsec:conditional_mppi}

TD-MPC2 では,
将来の行動系列を確率的にサンプリングし,
各軌道の累積コストに基づいて行動列を更新する
Model Predictive Path Integral(MPPI)制御
\cite{williams2015modelpredictivepathintegral}
が用いられている.
MPPI は,
勾配計算を必要とせず,
非線形・非凸なダイナミクスに対しても適用可能な
サンプリングベースのモデル予測制御手法である.

MPPI における行動更新は,
サンプルされた行動摂動 $\{\delta \mathbf{u}_t^{(k)}\}_{k=1}^K$ に対して,
各軌道のコスト $S^{(k)}$ を用いた指数重み付き平均として与えられる:
\begin{equation}
\mathbf{u}_t \leftarrow \mathbf{u}_t +
\frac{\sum_{k=1}^K \exp\!\left(-\frac{1}{\lambda} S^{(k)}\right)
\delta \mathbf{u}_t^{(k)}}
{\sum_{k=1}^K \exp\!\left(-\frac{1}{\lambda} S^{(k)}\right)} .
\end{equation}
ここで,$S^{(k)}$ は各行動系列に対する
ロールアウト結果に基づく累積コストであり,
将来状態の予測に依存する.

提案手法では,
このロールアウトに用いる世界モデルを,
推定された物理パラメータ $\hat{\boldsymbol{\theta}}_t$ で条件付けることで,
各サンプル軌道の状態遷移およびコスト評価を行う.
すなわち,
\begin{equation}
\mathbf{x}_{t+1} = f(\mathbf{x}_t, \mathbf{u}_t ; \hat{\boldsymbol{\theta}}_t)
\end{equation}
として予測された軌道に基づき,
$S^{(k)}$ を計算する.

この操作は,
MPPI の更新式における「コスト評価の部分」にのみ影響し,
サンプリング・重み付け・平均化という最適化構造そのものは変化しない.
したがって,
推定パラメータを条件として導入しても,
MPPI の理論的枠組みや最適化手順が破綻することはなく,
推定された環境モデルに整合した軌道計画が自然に実現される.

さらに,
MPPI は多数のサンプル軌道を並列に評価するため,
推定パラメータに誤差が含まれる場合でも,
単一のモデル予測に過度に依存することなく,
相対的に低コストな行動系列を選択できる.
この性質により,
提案手法は完全な同定が得られない状況においても,
安定した制御性能を維持することが可能である.


\section{学習アルゴリズム}
\label{sec:learning_algorithm}

本節では,
提案手法における学習アルゴリズムについて述べる.
提案手法は,
物理パラメータ推定器と,
条件付き TD-MPC2 コントローラという
異なる役割を持つ二つのモジュールから構成されている.
本研究では,
それぞれの目的および時間スケールの違いを考慮し,
両者を分離して学習する方針を採用する.


\subsection{物理パラメータ推定器の学習}
\label{subsec:estimator_training}

物理パラメータ推定器は,
相互作用履歴 \(\mathcal{H}_t\) を入力として,
環境に固有な物理パラメータ
\(\mathbf{c}_{\mathrm{phys}}\) を出力する回帰モデルとして構成される.
本研究では,
シミュレーション環境において
真の物理パラメータが既知であることを利用し,
推定器を教師あり学習によって訓練する.

具体的には,
履歴 \(\mathcal{H}_t\) と対応する真の物理パラメータ
\(\mathbf{c}_{\mathrm{phys}}\) の組を学習データとし,
次の二乗誤差損失を最小化する.

\begin{align}
\mathcal{L}_{\mathrm{est}}
=
\left\|
\hat{\mathbf{c}}_{\mathrm{phys},t}
-
\mathbf{c}_{\mathrm{phys}}
\right\|^2 .
\end{align}

ここで,
\(\hat{\mathbf{c}}_{\mathrm{phys},t}\) は
推定器による予測値を表す.
物理パラメータは,
単一エピソード内では一定とし,
履歴の任意の時刻において同一の教師信号を用いた.

\subsection{条件付き制御器の学習}
\label{subsec:controller_training}

条件付き制御器は,
推定された物理パラメータを条件として受け取り,
潜在空間上の世界モデルおよび価値関数を学習する.
学習アルゴリズムの基本構造は,
第\ref{chap:preliminary}章で述べた
標準的な TD-MPC2 と同一である.

具体的には,
観測 \(\mathbf{o}_t\) から得られた潜在状態
\(\mathbf{s}_t\) に対して,
条件付き遷移モデル
\(\mathbf{s}_{t+1} = f(\mathbf{s}_t, \mathbf{a}_t, \hat{\mathbf{c}}_{\mathrm{phys},t})\)
を用いた多段階予測を行い,
報酬モデルおよび価値関数を
TD誤差に基づいて更新する.

\subsection{勾配分離による学習安定化}
\label{subsec:gradient_decoupling}

提案手法では,
物理パラメータ推定器と
条件付き TD-MPC2 コントローラの間で,
勾配を明示的に分離する.
具体的には,
推定器の出力
\(\hat{\mathbf{c}}_{\mathrm{phys},t}\) を
制御器に入力する際に,
勾配を遮断する処理を行う.

この勾配分離の導入により,
制御タスクの達成度に基づく誤差信号が推定モジュールの学習
プロセスへと逆伝搬することを防ぐ.
これにより,学習初期の不安定な制御勾配が物理パラメータ
の推定精度を損なう事態を回避し,推定と制御という二つの学習プロセス
が相互に悪影響を及ぼしあいながら不安定化する事態を未然に防ぐ設計となっている.

提案手法では,
推定器は
「物理パラメータを推定する」という明確な目的のもとで学習され,
制御器は
「推定された条件下で最適な行動を選択する」
という役割に専念する.
この役割分担に基づく学習分離により,
全体として安定した学習および制御性能が得られると考えられる.

\chapter{実験設定と評価方法}
\label{chap:experiment}

本章では,第\ref{chap:method}章で提案した明示的物理パラメータ推定に基づく適応的制御手法の有効性を検証するために用いた,実験環境,評価タスク,比較手法,および評価指標について述べる.本章では実験条件と評価手順のみを記述し,結果の解釈および考察は次章に譲る.


\section{実験目的と検証仮説}
\label{sec:exp_objective}

本研究の目的は,物理パラメータが未知な環境において,明示的な同定に基づく条件付き計画を導入することで,MBRL+MPC における計画の信頼性と制御性能が向上することを示すことである.この目的に基づき,以下の仮説を検証する.

\begin{itemize}
\item \textbf{H1}: 提案手法は,静的な未知環境において,Non-adaptive TD-MPC2 と比較して性能劣化を抑制できる.
\item \textbf{H2}: 提案手法は,真の物理パラメータを与えた Oracle TD-MPC2 に近い性能を示す.
\item \textbf{H3}: 提案手法は,Domain Randomization と比較して,評価時の固定パラメータ環境におけるゼロショット汎化性能が高い,または同等の性能をより非保守的に達成する.
\item \textbf{H4}: 提案手法は,エピソード内で物理パラメータが変化する動的環境においても,変化に追従する挙動を示す.
\end{itemize}


\section{実装概要と再現性}
\label{sec:implementation_overview}

\subsection{シミュレーション基盤(DMControl)の概要}
\label{subsec:dmcontrol_overview}

本研究の実験は,MuJoCo 物理エンジンに基づく DeepMind Control Suite(DMControl)\cite{tassa2018deepmindcontrolsuite} を用いて行った.DMControl は,連続制御タスク群(例: Pendulum,Walker 等)を統一的な API で提供し,観測,行動,報酬の定義がタスクごとに明確に設計されている.本研究では,既存タスクをベースとして,物理パラメータの摂動を付与したタスク(Randomized / Fixed / Dynamic)を実装し,未知環境への適応性能を評価した.

\subsection{再現性のための共通設定}
\label{subsec:reproducibility}

すべての手法は同一コードベース上に実装し,比較手法間で計算手順や実装上の差異が生じないようにした.各実験は複数の乱数シードにより独立に実行し,統計的に安定した評価を行った.学習率,バッチサイズ,MPPI のサンプル数等のハイパーパラメータは Appendix(表\ref{tab:hyperparams})にまとめる.

\subsection{タスクごとの学習ステップ数}
\label{subsec:task_steps}

タスクごとにダイナミクスの複雑性や探索難易度が異なるため,本研究ではタスクごとに学習ステップ数を設定した.ただし,同一タスク内では,提案手法および比較手法が同一の学習ステップ数を共有することで,公平な比較を行った.タスク別の学習ステップ数,評価エピソード数,および実行設定を表\ref{tab:task_protocol}に示す.

\begin{table}[t]
\centering
\caption{タスク別の学習および評価プロトコル(例.数値はあなたの実験設定に合わせて記入)}
\label{tab:task_protocol}
\begin{tabular}{lccc}
\toprule
\textbf{Task} & \textbf{Train steps} & \textbf{Seeds}\\
\midrule
Pendulum-Swingup & \texttt{(50000)} & \texttt{(5)} \\
Walker-Walk  & \texttt{(100000)} & \texttt{(5)}  \\
\bottomrule
\end{tabular}
\end{table}


\section{評価環境とタスク設計}
\label{sec:task_design}

\subsection{共通の MDP 設定}
\label{subsec:mdp}

各タスクは連続制御問題として定式化される.時刻 $t$ の観測を $\mathbf{o}_t$,行動を $\mathbf{a}_t$ とし,環境からスカラー報酬 $r_t$ を得る.本研究では,DMControl が提供する観測定義をそのまま用い,観測の正規化は行わない.行動は各タスクで $[-1, 1]$ にスケーリングされた連続値である.


\subsection{評価タスク}
\label{subsec:evaluation_tasks}

\subsubsection{Walker-Walk タスク(基本設定)}
\label{subsubsec:walker_base}

Walker-Walk は,二次元平面上において二足歩行ロボットを前進歩行させる連続制御タスクである.ロボットは胴体(torso)および左右の股関節・膝関節・足首関節から構成され,動的バランスを維持しながら前進することが求められる.本研究では DMControl の \texttt{walker-walk} を用いる\cite{tassa2018deepmindcontrolsuite}.環境設定を表\ref{tab:walker_env}に示す.

\begin{table}[t]
\centering
\caption{Walker-Walk タスクの環境設定}
\label{tab:walker_env}
\begin{tabular}{lc}
\toprule
\textbf{項目} & \textbf{値} \\
\midrule
Task name & \texttt{walker-walk} \\
Control period & 0.025 s(40 Hz) \\
Episode length & 500 steps(action repeat=2 を含め 25 s 相当) \\
Termination & Timeout のみ(転倒しても継続) \\
\bottomrule
\end{tabular}
\end{table}

\paragraph{観測空間と行動空間}

Walker-Walk の観測および行動空間を表\ref{tab:walker_spaces}に示す.観測は姿勢($\sin,\cos$ 表現),胴体高さ,および速度(線速度・角速度)から構成される.行動は 6 つの関節トルクである.

\begin{table}[t]
\centering
\caption{Walker-Walk の観測空間と行動空間}
\label{tab:walker_spaces}
\begin{tabular}{lcl}
\toprule
\textbf{項目} & \textbf{次元} & \textbf{内容} \\
\midrule
観測 $\mathbf{o}_t$ & 24 & orientations(14), height(1), velocity(9) \\
行動 $\mathbf{a}_t$ & 6  & right/left hip,knee,ankle トルク \\
\bottomrule
\end{tabular}
\end{table}

\paragraph{報酬関数}

Walker-Walk の報酬は,立位の維持と前進速度を同時に促すよう設計されている.時刻 $t$ における報酬 $r_t$ は次式で定義される.
\begin{align}
r_t
=
\frac{
r_{\mathrm{stand}} \cdot (5 r_{\mathrm{move}} + 1)
}{6}.
\label{eq:walker_reward}
\end{align}

立位報酬 $r_{\mathrm{stand}}$ は胴体高さと上向き度に基づき,
\begin{align}
r_{\mathrm{stand}}
&=
\frac{3 r_{\mathrm{height}} + r_{\mathrm{upright}}}{4},
\\
r_{\mathrm{height}}
&=
\mathrm{tol}(h_t; 1.2, \infty),
\\
r_{\mathrm{upright}}
&=
\frac{1 + u_t}{2},
\end{align}
とする.ただし $h_t$ は胴体の高さ,$u_t \in [-1,1]$ は胴体の上向き度を表す.

移動報酬 $r_{\mathrm{move}}$ は水平方向速度 $v_t$ に基づき,
\begin{align}
r_{\mathrm{move}}
=
\mathrm{tol}(v_t; 1.0, \infty),
\end{align}
と定義する.ここで $\mathrm{tol}(\cdot)$ は DMControl における滑らかな tolerance 関数であり,指定した境界内では 1 をとり,境界から離れるほど連続的に 0 へ減衰する.本研究では,報酬定義を DMControl の実装に準拠する.


\subsubsection{Pendulum-Swingup タスク}
\label{subsubsec:pendulum}

Pendulum-Swingup は,倒立振子を下向き初期状態から振り上げ,上向き姿勢を維持する連続制御タスクである.本研究では DMControl の \texttt{pendulum-swingup} を用いる\cite{tassa2018deepmindcontrolsuite}.環境設定を表\ref{tab:pendulum_env}に示す.

\begin{table}[t]
\centering
\caption{Pendulum-Swingup タスクの環境設定}
\label{tab:pendulum_env}
\begin{tabular}{lc}
\toprule
\textbf{項目} & \textbf{値} \\
\midrule
Task name & \texttt{pendulum-swingup} \\
Control period & 0.02 s(50 Hz) \\
Episode length & 500 steps(action repeat=2 を含む) \\
Termination & Timeout のみ \\
\bottomrule
\end{tabular}
\end{table}

\paragraph{観測空間と行動空間}

観測は角度の $\cos,\sin$ および角速度から構成され,$\mathbf{o}_t \in \mathbb{R}^3$ である.行動は振子軸のトルクであり,$\mathbf{a}_t \in [-1,1]$ の 1 次元連続値である.

\paragraph{報酬関数}

Pendulum-Swingup の報酬は,上向き姿勢($\theta=0$)に近いほど高くなるよう設計されている.報酬は次式で与えられる.
\begin{align}
r_t
=
\mathrm{tol}(d_t; 0, 0),
\label{eq:pendulum_reward}
\end{align}
ただし $d_t$ は上向き位置からの角度距離を表す.DMControl の実装では,$d_t$ に対して cosine 形状の滑らかな tolerance が適用されるため,上向き($d_t=0$)で $r_t=1$,下向き($d_t \approx \pi$)で $r_t \approx 0$ となる.本研究では制御コスト項は含めない(DMControl 準拠).


\section{物理パラメータ摂動と評価条件}
\label{sec:perturbation_and_conditions}

本研究では,学習時に用いる訓練環境(Randomized)と,汎化評価のための評価環境(Fixed / Dynamic)を実装し,未知環境への適応性能を検証する.本節では,摂動パラメータと評価条件をまとめる.

\subsection{Domain Randomization(訓練環境)}
\label{subsec:dr_training}

Domain Randomization(DR)では,エピソードごとに物理パラメータをランダム化し,平均的に頑健な方策の学習を促す.本研究で用いたランダム化範囲を表\ref{tab:dr_ranges}に示す.いずれもエピソード内では固定とする.

\begin{table}[t]
\centering
\caption{DR におけるランダム化パラメータと範囲}
\label{tab:dr_ranges}
\begin{tabular}{lcc}
\toprule
\textbf{Task} & \textbf{Parameter} & \textbf{Range (per episode)} \\
\midrule
Walker-Walk (Actuator) & actuator gear scale & $0.4$--$1.4 \times$ default \\
Pendulum-Swingup & pendulum mass & $0.5$--$2.5$ kg \\
\bottomrule
\end{tabular}
\end{table}

\subsection{静的環境評価}
\label{subsec:fixed_eval}

学習済みモデルのゼロショット汎化性能を評価するため,物理パラメータを固定した評価環境を複数用意する.例として,Walker-Walk(質量)では torso 質量をスケールした複数条件を用意し,in-distribution(DR 範囲内)と out-of-distribution(範囲外)で性能を比較する.評価条件の一覧は Appendix(表\ref{tab:fixed_conditions})にまとめる.

\subsection{動的環境評価}
\label{subsec:dynamic_eval}

実環境では,バッテリー消耗や摩耗などにより,物理パラメータがエピソード内で時間変化する場合がある.そこで,Walker-Walk において actuator gear scale がエピソード内で線形に減衰する動的環境を用意する.本評価は,エピソード内でほぼ一定の物理環境を厳密に満たさない状況での挙動観察を目的とし,結果章では補助的な分析として扱う.


\section{比較手法}
\label{sec:baselines}

提案手法の有効性を検証するため,以下の比較手法を用いる.

\subsection{Non-adaptive TD-MPC2}
\label{subsec:baseline_nonadaptive}

推定器を用いず,標準の TD-MPC2 により制御を行う.未知環境における性能劣化の基準線として用いる.

\subsection{Domain Randomization}
\label{subsec:baseline_dr}

DR 訓練環境で学習したモデルを,Fixed / Dynamic 環境で評価する.頑健化戦略との比較対象とする.

\subsection{Oracle TD-MPC2}
\label{subsec:baseline_oracle}

真の物理パラメータを制御器に与えた TD-MPC2 を用いる.達成可能な上限性能を示す参照点として用いる.


\section{評価指標}
\label{sec:metrics}

制御性能の評価にはエピソード累積報酬を用いる.報告値として,外れ値に頑健な統計量である Interquartile Mean(IQM)を用いる\cite{agarwal2021deep}.必要に応じて,中央値および分位点も併記する(詳細は結果章で示す).
\chapter{実験結果}
\label{chap:results}

本章では,第\ref{chap:experiment}章で述べた実験設定に基づき,提案手法の有効性を検証した結果を示す.本章では実験結果のみを記述し,結果の解釈および考察は次章に譲る.


\section{物理パラメータ摂動に対する性能評価}
\label{sec:perturbation_performance}

本節では,物理パラメータを変化させた環境において,提案手法および比較手法の制御性能を評価した結果を示す.評価は,学習後のモデルを用いたゼロショット汎化性能として測定し,エピソード累積報酬により評価する.


\subsection{複数タスクにおける摂動評価}
\label{subsec:multi_task_performance}

図\ref{fig:performance_curves}に,複数のタスクおよび物理パラメータ条件における各手法の性能を示す.各サブプロットは異なるタスクまたは摂動条件を表し,横軸は物理パラメータの値,縦軸はエピソード累積報酬を表す.

\begin{figure}[tb]
\centering
\includegraphics[width=0.95\linewidth]{figures/performance_curves.png}
\caption{複数タスクにおける物理パラメータ摂動に対する性能評価}
\label{fig:performance_curves}
\end{figure}

図\ref{fig:performance_curves}より,提案手法(Adaptive TD-MPC2)は,複数のタスクおよび物理パラメータ条件において,Non-adaptive TD-MPC2 および Domain Randomization と比較して高い性能を示している.特に,Domain Randomization の訓練範囲外(out-of-distribution)の条件においても,性能劣化が抑制されていることが確認できる.

また,Oracle TD-MPC2(真の物理パラメータを与えた場合)に近い性能を達成しており,推定器による適応的計画が有効に機能していることが示される.


\subsection{Pendulum-Swingup における質量摂動評価}
\label{subsec:pendulum_performance}

Pendulum-Swingup タスクにおいて,振子の質量を変化させた環境での性能評価結果を図\ref{fig:pendulum_estimation_convergence}に示す.

\begin{figure}[tb]
\centering
\includegraphics[width=0.85\linewidth]{figures/pendulum_estimation_convergence.png}
\caption{Pendulum-Swingup における質量摂動に対する性能評価}
\label{fig:pendulum_estimation_convergence}
\end{figure}

図\ref{fig:pendulum_estimation_convergence}より,提案手法は質量パラメータの広範な範囲において安定した性能を示している.Non-adaptive TD-MPC2 は特定の質量条件下で性能が低下するのに対し,提案手法は適応的計画により性能を維持していることが確認できる.


\subsection{Walker-Walk におけるアクチュエータ摂動評価}
\label{subsec:walker_actuator_performance}

Walker-Walk タスクにおいて,アクチュエータの出力スケールを変化させた環境での性能評価結果を示す.

\subsubsection{エピソード内での適応挙動}
\label{subsubsec:walker_estimation_convergence}

図\ref{fig:walker_estimation_convergence}に,エピソード内での各手法の適応挙動を示す.

\begin{figure}[tb]
\centering
\includegraphics[width=0.85\linewidth]{figures/walker_estimation_convergence.png}
\caption{Walker-Walk におけるエピソード内での適応挙動}
\label{fig:walker_estimation_convergence}
\end{figure}

図\ref{fig:walker_estimation_convergence}より,提案手法はエピソード内で相互作用が進むにつれて性能が向上する傾向が見られる.これに対し,Non-adaptive TD-MPC2 は一定の性能に留まっており,推定に基づく適応的計画の効果が示唆される.


\subsubsection{アクチュエータスケール変化に対する性能}
\label{subsubsec:walker_actuator_scale}

図\ref{fig:walker_actuator_performance}に,異なる actuator gear scale 条件下での各手法の性能を示す.横軸は actuator gear scale の値,縦軸はエピソード累積報酬を表す.

\begin{figure}[tb]
\centering
\includegraphics[width=0.9\linewidth]{figures/walker_actuator_performance_curves.png}
\caption{Walker-Walk における actuator gear scale 変化に対する性能評価}
\label{fig:walker_actuator_performance}
\end{figure}

図\ref{fig:walker_actuator_performance}より,提案手法は Domain Randomization の訓練範囲(\(0.4\)--\(1.4\) 倍)内外において,Non-adaptive TD-MPC2 よりも高い性能を示している.特に,範囲外の条件においても性能劣化が抑制されており,推定に基づく適応的計画が有効に機能していることが確認できる.

また,提案手法は Oracle TD-MPC2 に近い性能を達成しており,物理パラメータの明示的同定が制御性能の向上に寄与していることが示される.一方,Domain Randomization は訓練範囲内では頑健性を示すものの,範囲外では性能が低下する傾向が見られる.


\section{結果のまとめ}
\label{sec:results_summary}

本章で示した実験結果を以下にまとめる.

\paragraph{摂動評価における性能}

Pendulum-Swingup および Walker-Walk の両タスクにおいて,提案手法(Adaptive TD-MPC2)は,物理パラメータの広範な摂動条件下で安定した性能を示した.特に,Domain Randomization の訓練範囲外(out-of-distribution)においても,Non-adaptive TD-MPC2 と比較して性能劣化が抑制されていることが確認された.


\paragraph{比較手法との関係}

Non-adaptive TD-MPC2 は未知環境において性能劣化を示し,Domain Randomization は訓練範囲内では頑健性を示すものの範囲外での性能低下が見られた.これに対し,提案手法は Oracle TD-MPC2(真の物理パラメータを与えた場合)に近い性能を達成しており,推定に基づく適応的計画が有効に機能していることが示された.


\paragraph{エピソード内での適応}

Walker-Walk タスクにおいて,提案手法はエピソード内で相互作用が進むにつれて性能が向上する傾向が確認された.これは,相互作用履歴から物理パラメータを推定し,計画に反映することで適応的な制御が実現されていることを示唆する.

\chapter{考察}
\label{chap:discussion}

本章では,第\ref{chap:results}章で示した実験結果に対する解釈および考察を行う.
各手法の性能差が生じた要因を分析し,
提案手法の有効性と限界について議論する.


\section{物理パラメータ摂動に対する性能差の解釈}
\label{sec:performance_difference_interpretation}

本節では,物理パラメータが変化した環境における各手法の性能差について,
その要因を分析する.


\subsection{Pendulum-Swingup における手法間性能差}
\label{subsec:pendulum_performance_discussion}

図\ref{fig:performance_curves}および表\ref{tab:pendulum_results}に示したように,
Pendulum-Swingup タスクにおいて,
Domain Randomization(DR)は提案手法(Adaptive)を上回る性能を示した.
この結果は,タスクの特性と各手法の適応メカニズムの相互作用によって説明される.

振子の質量が増加すると,
同じ角度変化を実現するために必要なトルクが増大する.
このため,質量パラメータの変化に対する適応には,
より大きな出力を生成する制御則が求められる.
Domain Randomization では,
訓練時に質量パラメータを広範囲にわたってランダム化することで,
多様なトルク要求に対応可能なロバストな方策を獲得している.
その結果,テスト時に質量が変化した環境においても,
既に経験した範囲内であれば高い性能を維持できる.

これに対し,提案手法(Adaptive)は,
エピソード内で物理パラメータをオンライン推定し,
推定値に基づいて動力学モデルを適応させる.
図\ref{fig:pendulum_estimation_convergence}に示されるように,
推定値はエピソード初期に大きく変動し,
時間の経過とともに真値へ収束する挙動を示す.
この収束過程において,
推定値が真値から乖離している期間では,
動力学モデルの予測精度が低下し,
制御性能が一時的に劣化する.
特に,エピソード初期の推定ラグは累積報酬の低下に直接影響するため,
Pendulum のような比較的単純なタスクにおいては,
推定ラグのコストが適応の利益を上回る結果となったと考えられる.

一方,Oracle TD-MPC2 は,
すべての質量条件において最も高い性能を示している.
これは,真の物理パラメータを初めから利用できるため,
推定ラグが存在せず,
エピソード全体を通して最適な動力学モデルに基づく制御が可能であることに起因する.
Oracle の性能は,
物理パラメータが正確に既知である場合の性能上限を示しており,
提案手法における推定精度の向上が性能改善の鍵となることを示唆している.


\subsection{Walker-Walk における手法間性能差}
\label{subsec:walker_performance_discussion}

図\ref{fig:walker_actuator_performance}および表\ref{tab:walker_results}に示したように,
Walker-Walk タスクにおいては,
提案手法(Adaptive)が平均的に最も高い性能を示し,
特に低アクチュエータスケール条件において顕著な優位性が確認された.
この結果は,タスクの複雑性と適応メカニズムの有効性の関係を示唆している.

Walker-Walk タスクは,
6自由度のアクチュエータを協調的に制御する必要があり,
Pendulum-Swingup と比較して状態空間および行動空間の次元が高い.
このような高次元タスクにおいては,
Domain Randomization により獲得される方策は,
広範なパラメータ範囲に対するロバスト性を獲得する一方で,
特定のパラメータ条件に対する最適性は犠牲になる可能性がある.

これに対し,提案手法は,
エピソード内でアクチュエータスケールを推定し,
推定値に基づいて動力学モデルを調整することで,
各環境条件に特化した制御を実現できる.
特に,アクチュエータスケールが小さい条件では,
出力制約が厳しくなるため,
正確な動力学モデルに基づく最適化の重要性が増す.
提案手法は,この条件下において,
推定に基づく適応により Non-adaptive および DR を大きく上回る性能を達成している.

% また,表\ref{tab:walker_results}において,
% アクチュエータスケールが訓練条件($1.0\times$)から離れるほど,
% 提案手法と他手法の性能差が拡大する傾向が観測される.
% これは,訓練範囲外の条件において,
% オンライン適応の利益がより顕著に現れることを示している.


\section{動的環境における適応性能の要因分析}
\label{sec:dynamic_adaptation_analysis}

図\ref{fig:walker_dynamic_compare}に示したように,
アクチュエータスケールがエピソード内で時間的に変化する動的環境において,
提案手法は他手法と比較して高い報酬を維持した.
本節では,この動的適応性能の要因を分析する.

エピソード内でアクチュエータスケールが線形に減衰する環境では,
固定された方策を用いる Non-adaptive TD-MPC2 は,
時間の経過とともに環境との不整合が拡大し,
性能が徐々に低下する.
Domain Randomization は,
訓練時に多様なスケール条件を経験しているため,
Non-adaptive よりも緩やかな性能低下を示すが,
エピソード後半では依然として劣化が観測される.

提案手法は,図\ref{fig:walker_dynamic_estimation}に示されるように,
時間とともに変化するアクチュエータスケールを追従的に推定する能力を持つ.
推定値は真値の変化に遅れながらも同様の減少傾向を示しており,
この追従能力により,
エピソード後半においても動力学モデルと実環境の乖離を抑制できる.
その結果,図\ref{fig:walker_dynamic_compare}の下段に示されるように,
提案手法は後半まで高い報酬を維持し,
他手法よりも小さい劣化に留まっている.

この結果は,提案手法が静的な環境変化だけでなく,
動的に変化する環境に対しても適応可能であることを示しており,
実世界の経年劣化や連続的な環境変動への対応において有用性が高いと考えられる.


\section{推定挙動と制御性能の関係}
\label{sec:estimation_control_relationship}

図\ref{fig:pendulum_estimation_convergence}および図\ref{fig:walker_estimation_convergence}に示されるように,
提案手法における物理パラメータの推定値は,
エピソード初期に大きく変動した後,
時間の経過とともに真値付近へ収束する挙動を示す.
本節では,この推定挙動と制御性能の関係について考察する.

推定値がエピソード初期に変動する理由として,
初期状態における観測データの不足が挙げられる.
物理パラメータ推定器は,
観測された状態遷移の履歴に基づいて推定を行うため,
エピソード開始直後は十分な情報が蓄積されておらず,
推定値は不確実性が高い状態にある.
相互作用が進行し観測データが蓄積されるにつれて,
推定値は真値に近づき,安定化する.

興味深いことに,図\ref{fig:walker_actuator_performance}および図\ref{fig:walker_dynamic_compare}に示されるように,
推定値が完全に収束していない状態においても,
提案手法は高い制御性能を維持している.
これは,TD-MPC2 のモデル予測制御が,
ある程度の動力学モデル誤差に対してロバストであることを示唆している.
モデル予測制御では,
現在の状態から短い時間窓で軌道最適化を行い,
その結果得られた行動のうち最初の一歩のみを実行する.
この再計画のサイクルにより,
モデル誤差の累積が抑制され,
推定値に多少の誤差が含まれていても,
実用的な制御性能が得られると考えられる.

ただし,推定誤差が大きい場合や,
推定値が真値から大きく乖離している期間が長い場合には,
制御性能の低下が避けられない.


\section{提案手法の有効性}
\label{sec:effectiveness}

本研究の実験結果を総合すると,
提案手法は以下の条件において有効性が高いと考えられる:

\begin{itemize}
\item \textbf{高次元・複雑なタスク}:
Walker-Walk のような高次元タスクでは,
オンライン適応の利益が推定ラグのコストを上回る.

\item \textbf{訓練範囲外の大きな摂動}:
パラメータが訓練範囲から大きく逸脱する条件では,
Domain Randomization のロバスト性では対応しきれず,
適応的な推定が有効となる.

\item \textbf{動的に変化する環境}:
エピソード内でパラメータが時間変化する場合,
固定方策では対応困難であり,
追従的な推定が必要となる.
\end{itemize}

これらの知見は,
実環境へのモデルベース強化学習の展開において,
環境特性の明示的な推定が有効な適応戦略となりうることを示している.

\chapter{結論}
\label{chap:conclusion}

\section{研究の総括}
\label{sec:summary}

本研究では,
物理パラメータが未知な環境において,
モデルベース強化学習における計画性能を維持するための
適応的制御手法を提案した.

既存のモデルベース強化学習手法は,
学習時に想定された物理パラメータから
実行時の環境が乖離した場合,
世界モデルの予測精度が低下し,
計画の信頼性が損なわれるという課題を有していた.

これに対し本研究では,
環境の物理パラメータを過去の相互作用履歴から明示的に推定し,
その推定結果を世界モデルおよび計画過程に条件として与える枠組みを構築した.
推定と制御を概念的に分離する設計により,
学習の安定性を保ちつつ環境変動への適応能力を付加することが可能となった.

Pendulum-Swingup および Walker-Walk タスクを用いた実験により,
提案手法は高次元タスクにおいて Non-adaptive および DRモデル を上回る性能を達成し,
動的環境においても変化に追従する挙動を示すことが確認された.
また,真の物理パラメータを与えた Oracle TD-MPC2 が
環境変動下でも高い性能を維持することを実証し,
物理パラメータの明示的な推定が計画の信頼性向上に寄与することを示した.


\section{今後の課題}
\label{sec:future_work}

本研究の提案手法には以下の課題が残されている.

\paragraph{推定ラグの影響}
エピソード初期における推定値の収束遅延は,
特に短時間タスクや単純なタスクにおいて累積報酬を低下させる.
推定の収束速度向上のため,
初期化戦略や推定器アーキテクチャの改良が必要である.

\paragraph{多次元パラメータへのスケーラビリティ}
本研究では単一パラメータの推定に焦点を当てたが,
実環境では複数の物理パラメータが同時に変化する可能性がある.
多次元パラメータ推定における収束性や,
推定すべきパラメータの優先順位付けの検討が必要である.

\paragraph{実機ロボットへの展開}
シミュレーション環境での検証にとどまっており,
実機展開においては
Sim2Real ギャップ,観測ノイズ,アクチュエータ遅延などへの対処が必要となる.
また,真の物理パラメータへのアクセスが困難な実環境における
推定器の学習戦略の再検討も重要な課題である.

\paragraph{計算コストの削減}
GRU による系列推論は追加の計算コストを伴うため,
リアルタイム制約下での実用性向上のため,
推定器の軽量化や推定頻度の最適化が求められる.

これらの課題に取り組むことで,
実環境における不確実性に対してより頑健かつ適応的なロボット制御システムの実現が期待される.


\begin{acknowledgements}
% 謝辞
本研究を進めるに際して, 指導教員の石井信教授, 国
際電気通信基礎技術研究所 (ATR) の久保顕大さんから多くの助言, 指導を頂きま
した. 厚く感謝を申し上げます. また, 研究の準備および議論を通じて協力して頂
いた石井研究室の皆様にも感謝申し上げます.
\end{acknowledgements}

\clearpage
\appendix
\chapter{Experimental Details}

\section{Hyperparameter Settings}
% tables/hyperparameters_tdmpc2.tex
% Requires: \usepackage{booktabs}

\begin{table}[htbp]
  \centering
  \caption{Hyperparameter settings. We directly apply settings in~\cite{hansen2023td} for the shared hyperparameters without further tuning. We share the same setting across all tasks demonstrated before.}
  \label{tab:hyperparameters}
  \footnotesize
  \setlength{\tabcolsep}{6pt}
  \renewcommand{\arraystretch}{1.15}

  \begin{tabular}{l|c}
    \toprule
    \textbf{Hyperparameter} & \textbf{Value} \\
    \midrule

    \multicolumn{2}{l}{\textbf{Training}} \\
    \midrule
    Learning rate & $3 \times 10^{-4}$ \\
    Batch size & 256 \\
    Buffer size & 1,000,000 \\
    Sampling & Uniform \\
    Reward loss coefficient ($c_r$) & 0.1 \\
    Value loss coefficient ($c_q$) & 0.1 \\
    Consistency loss coefficient ($c_d$) & 20 \\
    Discount factor ($\gamma$) & 0.99 \\
    Target network update rate & 0.5 \\
    Gradient Clipping Norm & 20 \\
    Optimizer & Adam \\
    \midrule

    \multicolumn{2}{l}{\textbf{Planner}} \\
    \midrule
    MPPI Iterations & 6 \\
    Number of samples & 512 \\
    Number of elites & 64 \\
    Number policy rollouts & 24 \\
    horizon & 3 \\
    Minimum planner std & 0.05 \\
    Maximum planner std & 2 \\
    \midrule

    \multicolumn{2}{l}{\textbf{Actor}} \\
    \midrule
    Minimum policy log std & $-10$ \\
    Maximum policy log std & 2 \\
    Entropy coefficient ($\alpha$) & $1 \times 10^{-4}$ \\
    Prior constraint coefficient ($\beta$) & 1.0 \\
    Scale Threshold ($s$) & 2.0 \\
    \midrule

    \multicolumn{2}{l}{\textbf{Critic}} \\
    \midrule
    Q functions Ensemble & 5 \\
    Number of bins & 101 \\
    Minimum value & $-10$ \\
    Maximum value & 10 \\
    \midrule

    \multicolumn{2}{l}{\textbf{Architecture (5M)}} \\
    \midrule
    Encoder layers & 2 \\
    Encoder dimension & 256 \\
    MLP hidden layer dimension & 512 \\
    Latent space dimension & 512 \\
    Task embedding dimension & 96 \\
    Q function drop out rate & 0.01 \\
    MLP activation & Mish \\
    MLP Normalization & LayerNorm \\
    \bottomrule
  \end{tabular}
\end{table}

\clearpage

\section{Task Specifications}
\label{sec:appendix_task_specs}

本節では,第\ref{chap:experiment}章で用いた評価タスクの詳細仕様について述べる.

\subsection{Pendulum-Swingup タスク仕様}
\label{subsec:appendix_pendulum}

\subsubsection{環境パラメータ}

\begin{table}[h]
\centering
\caption{Pendulum-Swingup 環境パラメータ}
\label{tab:appendix_pendulum_params}
\begin{tabular}{lc}
\toprule
\textbf{パラメータ} & \textbf{値} \\
\midrule
制御周期 & 0.02 s (50 Hz) \\
Action repeat & 2 \\
エピソード長 & 500 steps \\
観測次元 & 3 \\
行動次元 & 1 \\
\midrule
\multicolumn{2}{l}{\textit{物理パラメータ(デフォルト)}} \\
振子質量 & 1.0 kg \\
関節damping & 0.001 \\
重力加速度 & $-9.81$ m/s$^2$ \\
タイムステップ & 0.01 s \\
\bottomrule
\end{tabular}
\end{table}

\subsubsection{観測空間}

観測 $\mathbf{o}_t \in \mathbb{R}^3$ は以下から構成される:
\begin{itemize}
\item \textbf{orientation} (2次元): 振子角度 $\theta$ の $(\cos\theta, \sin\theta)$ 表現
\item \textbf{velocity} (1次元): 角速度 $\dot{\theta}$ [rad/s]
\end{itemize}

\subsubsection{行動空間}

行動 $\mathbf{a}_t \in [-1, 1]$ は振子軸に加えるトルクを表す1次元連続値である.

\subsubsection{報酬関数}

時刻 $t$ における報酬 $r_t$ は,上向き位置からの角度距離 $d_t$ に基づき,
\begin{align}
r_t = \mathrm{tolerance}(d_t; \text{bounds}=(0, 0), \text{margin}=\pi, \text{sigmoid}=\text{`cosine'}),
\end{align}
と定義される.ここで,$\mathrm{tolerance}(\cdot)$ は DMControl における滑らかな許容関数であり,
上向き位置 ($d_t=0$) で $r_t=1$,下向き位置 ($d_t \approx \pi$) で $r_t \approx 0$ となる.
報酬範囲は $[0, 1]$ である.

\subsubsection{ランダム化設定}

Domain Randomization 訓練では,エピソードごとに振子質量を一様分布 $\mathcal{U}(0.5, 2.5)$ kg からサンプリングする.


\subsection{Walker-Walk タスク仕様}
\label{subsec:appendix_walker}

\subsubsection{環境パラメータ}

\begin{table}[h]
\centering
\caption{Walker-Walk 環境パラメータ}
\label{tab:appendix_walker_params}
\begin{tabular}{lc}
\toprule
\textbf{パラメータ} & \textbf{値} \\
\midrule
制御周期 & 0.025 s (40 Hz) \\
Action repeat & 2 \\
エピソード長 & 500 steps (25 s 相当) \\
観測次元 & 24 \\
行動次元 & 6 \\
\midrule
\multicolumn{2}{l}{\textit{物理パラメータ(デフォルト)}} \\
Torso 質量 & 3.34 kg \\
床摩擦係数 & [0.7, 0.1, 0.1] \\
関節damping & 0.1 \\
関節armature & 0.01 \\
重力加速度 & $-9.81$ m/s$^2$ \\
タイムステップ & 0.0025 s \\
\bottomrule
\end{tabular}
\end{table}

\subsubsection{観測空間}

観測 $\mathbf{o}_t \in \mathbb{R}^{24}$ は以下から構成される:
\begin{itemize}
\item \textbf{orientations} (14次元): 胴体および各関節の姿勢($\sin$, $\cos$ 表現)
\item \textbf{height} (1次元): 胴体の地面からの高さ
\item \textbf{velocity} (9次元): 胴体の線速度・角速度,および各関節の角速度
\end{itemize}

\subsubsection{行動空間}

行動 $\mathbf{a}_t \in [-1, 1]^6$ は,左右の股関節・膝関節・足首関節(hip, knee, ankle)に加えるトルクを表す6次元連続値である.

\begin{table}[h]
\centering
\caption{Walker-Walk の行動空間}
\label{tab:appendix_walker_action}
\begin{tabular}{clcc}
\toprule
\textbf{Index} & \textbf{関節名} & \textbf{デフォルトgear} & \textbf{範囲} \\
\midrule
0 & right\_hip & 100 & $[-1, 1]$ \\
1 & right\_knee & 50 & $[-1, 1]$ \\
2 & right\_ankle & 20 & $[-1, 1]$ \\
3 & left\_hip & 100 & $[-1, 1]$ \\
4 & left\_knee & 50 & $[-1, 1]$ \\
5 & left\_ankle & 20 & $[-1, 1]$ \\
\bottomrule
\end{tabular}
\end{table}

実際に環境に加わるトルクは,行動値 $a_i$ と対応する gear 値 $g_i$ の積 $a_i \times g_i$ として計算される.

\subsubsection{報酬関数}

時刻 $t$ における報酬 $r_t$ は,立位報酬 $r_{\mathrm{stand}}$ と移動報酬 $r_{\mathrm{move}}$ の組み合わせとして,
\begin{align}
r_t = \frac{r_{\mathrm{stand}} \cdot (5 r_{\mathrm{move}} + 1)}{6},
\end{align}
と定義される.

立位報酬は,胴体高さ $h_t$ と上向き度 $u_t \in [-1,1]$ に基づき,
\begin{align}
r_{\mathrm{stand}} &= \frac{3 r_{\mathrm{height}} + r_{\mathrm{upright}}}{4}, \\
r_{\mathrm{height}} &= \mathrm{tolerance}(h_t; \text{bounds}=(1.2, \infty), \text{margin}=0.6), \\
r_{\mathrm{upright}} &= \frac{1 + u_t}{2},
\end{align}
とする.

移動報酬は,水平方向速度 $v_t$ に基づき,
\begin{align}
r_{\mathrm{move}} = \mathrm{tolerance}(v_t; \text{bounds}=(1.0, \infty), \text{margin}=0.5, \text{sigmoid}=\text{`linear'}),
\end{align}
と定義する.報酬範囲は理論的には $[0, 1]$ である.

\subsubsection{ランダム化設定}

本研究では,Walker-Walk において2種類の物理パラメータ摂動を用いた:

\paragraph{Torso 質量のランダム化}
エピソードごとに Torso 質量を一様分布 $\mathcal{U}(0.5 \times 3.34, 2.5 \times 3.34)$ kg = $\mathcal{U}(1.67, 8.35)$ kg からサンプリングする.これは荷物運搬やペイロード変化を模擬する.

\paragraph{Actuator gear のランダム化}
エピソードごとに全アクチュエータの gear 値に対して,共通のスケール $s \sim \mathcal{U}(0.4, 1.4)$ を乗じる.これはモーター劣化やバッテリー消耗を模擬する.

\subsubsection{動的環境設定}

動的環境評価では,Walker-Walk において actuator gear スケールがエピソード内で線形に減衰する設定を用いた.初期値 $s_0=1.0$ から最終値 $s_{T}=0.4$ まで,500ステップにわたり線形減衰させる:
\begin{align}
s_t = 1.0 + (0.4 - 1.0) \times \frac{t}{T-1},
\end{align}
ただし $T=500$ である.この設定は,バッテリーの連続消耗などを模擬する.


\clearpage
\section{Implementation Details}
\label{sec:appendix_implementation}

本節では,第\ref{chap:proposal}章で提案した手法の実装上の詳細について述べる.

\subsection{システムアーキテクチャ}
\label{subsec:appendix_system_architecture}

提案手法は,TD-MPC2 に明示的物理パラメータ推定モジュールを追加し,推定結果を条件として計画および制御に反映する構成をとる.システム全体は以下の2つのフェーズから構成される:

\paragraph{フェーズ1: 物理パラメータ推定}
過去 $K$ ステップ分の観測–行動履歴 $(\mathbf{o}_{t-K:t}, \mathbf{a}_{t-K:t-1})$ を GRU エンコーダに入力し,物理パラメータの潜在表現 $\hat{\mathbf{c}}_{\mathrm{phys}}$ を推定する.

\paragraph{フェーズ2: 条件付き計画}
推定された $\hat{\mathbf{c}}_{\mathrm{phys}}$ を世界モデル(dynamics, reward, Q-functions, policy prior)の条件として与え,MPPI プランナーにより行動系列を最適化する.

\paragraph{勾配の分離}
推定器とプランナーの学習を独立に保つため,推定された $\hat{\mathbf{c}}_{\mathrm{phys}}$ をプランナーに渡す際に勾配を切断する(\texttt{.detach()}).これにより,プランナーの学習が推定器を不安定化することを防ぐ.


\subsection{物理パラメータ推定器}
\label{subsec:appendix_physics_estimator}

\subsubsection{GRU アーキテクチャ}

物理パラメータ推定器は,観測–行動履歴から物理パラメータを回帰する GRU ベースのニューラルネットワークである.

\paragraph{入力}
時系列 $\{(\mathbf{o}_{\tau}, \mathbf{a}_{\tau})\}_{\tau=t-K}^{t}$,ただし $K=50$ をコンテキスト長とする.

\paragraph{アーキテクチャ}
\begin{enumerate}
\item \textbf{Input Projection}: 観測と行動を連結し,隠れ次元 256 に線形変換
\begin{align}
\mathbf{h}_{\tau}^{(0)} = \mathrm{ReLU}(\mathrm{LayerNorm}(\mathbf{W}_{\mathrm{in}} [\mathbf{o}_{\tau}; \mathbf{a}_{\tau}]))
\end{align}

\item \textbf{GRU}: 2層,隠れ次元 256,dropout 0.1
\begin{align}
\mathbf{h}_{\tau}^{(l)} = \mathrm{GRU}^{(l)}(\mathbf{h}_{\tau}^{(l-1)}), \quad l=1,2
\end{align}

\item \textbf{Output Head}: 最終ステップの隠れ状態から物理パラメータを回帰
\begin{align}
\hat{\mathbf{c}}_{\mathrm{phys}} = \tanh(\mathbf{W}_{\mathrm{out2}} \, \mathrm{ReLU}(\mathrm{LayerNorm}(\mathbf{W}_{\mathrm{out1}} \mathbf{h}_{t}^{(2)})))
\end{align}
\end{enumerate}

出力は $\hat{\mathbf{c}}_{\mathrm{phys}} \in [-1, 1]^{d_c}$ の範囲に正規化される.本研究では $d_c=8$ を用いた.

\paragraph{損失関数}
推定器は,真の物理パラメータ $\mathbf{c}_{\mathrm{phys}}^{\mathrm{true}}$ との平均二乗誤差により学習する:
\begin{align}
\mathcal{L}_{\mathrm{aux}} = \|\hat{\mathbf{c}}_{\mathrm{phys}} - \mathbf{c}_{\mathrm{phys}}^{\mathrm{true}}\|_2^2
\end{align}

推定器は独立した Optimizer(Adam, 学習率 $3 \times 10^{-4}$)により更新される.


\subsubsection{物理パラメータの正規化}

環境ラッパー(\texttt{PhysicsParamWrapper})が真の物理パラメータを取得し,以下のように正規化する:
\begin{align}
\mathbf{c}_{\mathrm{phys}}^{\mathrm{normalized}} = \frac{\mathbf{c}_{\mathrm{phys}}^{\mathrm{raw}} - \mathbf{c}_{\mathrm{default}}}{\sigma} \in [-1, 1]
\end{align}
ここで,$\mathbf{c}_{\mathrm{default}}$ はデフォルト値,$\sigma$ はランダム化範囲に基づくスケールである.


\subsection{条件付き世界モデル}
\label{subsec:appendix_conditional_world_model}

提案手法の世界モデルは,推定された物理パラメータ $\hat{\mathbf{c}}_{\mathrm{phys}}$ を条件として受け取る.ベースラインの TD-MPC2 と比較して,各モジュールの入力に $\hat{\mathbf{c}}_{\mathrm{phys}}$ を追加する.

\subsubsection{条件付けの実装}

各ネットワークの入力に物理パラメータを連結する:

\paragraph{Dynamics Model}
\begin{align}
\mathbf{z}_{t+1} = f_{\mathrm{dyn}}([\mathbf{z}_t; \mathbf{a}_t; \mathbf{e}_{\mathrm{task}}; \hat{\mathbf{c}}_{\mathrm{phys}}])
\end{align}

\paragraph{Reward Model}
\begin{align}
\hat{r}_t = f_{\mathrm{rew}}([\mathbf{z}_t; \mathbf{a}_t; \mathbf{e}_{\mathrm{task}}; \hat{\mathbf{c}}_{\mathrm{phys}}])
\end{align}

\paragraph{Q-functions}
\begin{align}
Q^{(i)}(\mathbf{z}_t, \mathbf{a}_t) = f_Q^{(i)}([\mathbf{z}_t; \mathbf{a}_t; \mathbf{e}_{\mathrm{task}}; \hat{\mathbf{c}}_{\mathrm{phys}}]), \quad i=1,\ldots,5
\end{align}

\paragraph{Policy Prior}
\begin{align}
\pi(\mathbf{a}_t | \mathbf{z}_t) = f_{\pi}([\mathbf{z}_t; \mathbf{e}_{\mathrm{task}}; \hat{\mathbf{c}}_{\mathrm{phys}}])
\end{align}

ここで,$[\cdot;\cdot]$ はベクトルの連結を表す.$\mathbf{e}_{\mathrm{task}}$ はタスク埋め込み(マルチタスク設定の場合)である.


\subsection{学習アルゴリズム}
\label{subsec:appendix_training_algorithm}

提案手法の学習は,推定器の学習(教師あり)とプランナーの学習(強化学習)の2フェーズから構成される.

\subsubsection{全体の学習手順}

各ステップにおいて,以下の手順で学習を行う:

\begin{enumerate}
\item \textbf{経験の収集}: 環境とインタラクトし,観測 $\mathbf{o}_t$,行動 $\mathbf{a}_t$,報酬 $r_t$,および真の物理パラメータ $\mathbf{c}_{\mathrm{phys}}^{\mathrm{true}}$ を記録

\item \textbf{Buffer への保存}: 過去 $K$ ステップの履歴とともに経験を保存

\item \textbf{バッチサンプリング}: Buffer から $(H+1)$ ステップのトラジェクトリをサンプル

\item \textbf{推定器の更新}: 
\begin{align}
\theta_{\mathrm{GRU}} \leftarrow \theta_{\mathrm{GRU}} - \alpha_{\mathrm{GRU}} \nabla_{\theta_{\mathrm{GRU}}} \mathcal{L}_{\mathrm{aux}}
\end{align}

\item \textbf{物理パラメータの推定}: 
\begin{align}
\hat{\mathbf{c}}_{\mathrm{phys}} = f_{\mathrm{GRU}}(\{\mathbf{o}_{\tau}, \mathbf{a}_{\tau}\}_{\tau=t-K}^{t}; \theta_{\mathrm{GRU}})
\end{align}

\item \textbf{勾配の切断}: 
\begin{align}
\hat{\mathbf{c}}_{\mathrm{phys}} \leftarrow \mathrm{detach}(\hat{\mathbf{c}}_{\mathrm{phys}})
\end{align}

\item \textbf{プランナーの更新}: $\hat{\mathbf{c}}_{\mathrm{phys}}$ を条件として,TD-MPC2 の損失により世界モデルを更新
\begin{align}
\mathcal{L}_{\mathrm{TDMPC2}} &= \mathcal{L}_{\mathrm{consistency}} + \mathcal{L}_{\mathrm{reward}} + \mathcal{L}_{\mathrm{value}} + \mathcal{L}_{\mathrm{termination}} \\
\theta_{\mathrm{planner}} &\leftarrow \theta_{\mathrm{planner}} - \alpha_{\mathrm{planner}} \nabla_{\theta_{\mathrm{planner}}} \mathcal{L}_{\mathrm{TDMPC2}}
\end{align}
\end{enumerate}

重要な点として,ステップ6の勾配切断により,プランナーの勾配が推定器に逆伝播しない.これにより,各モジュールが独立に学習可能となる.


\subsubsection{Optimizer の設定}

提案手法では3つの独立した Optimizer を用いる:

\begin{table}[h]
\centering
\caption{Optimizer の設定}
\label{tab:appendix_optimizers}
\begin{tabular}{lcc}
\toprule
\textbf{Optimizer} & \textbf{対象パラメータ} & \textbf{学習率} \\
\midrule
GRU Optimizer & 推定器 & $3 \times 10^{-4}$ \\
Planner Optimizer & Encoder, Dynamics, Reward, Q, Termination & $1 \times 10^{-3}$ \\
Policy Optimizer & Policy Prior & $1 \times 10^{-3}$ \\
\bottomrule
\end{tabular}
\end{table}


\subsubsection{教師ラベルの取得}

シミュレーション環境では,真の物理パラメータは環境ラッパーから直接取得可能である.各エピソードの開始時に物理パラメータがサンプリングされ,エピソード中は固定される.この物理パラメータを正規化したものを教師ラベルとして用いる.

実世界への展開においては,キャリブレーションや設計値を教師ラベルとして利用することが考えられる.また,オンライン同定手法(例:最小二乗法)により推定した値を擬似ラベルとして用いることも可能である.


% ===== bibliography =====
% 1) bst を refs/ に置く場合
\bibliographystyle{refs/kueethesis}
% 2) bib を refs/ に置く場合(例:refs/main.bib)

\bibliography{refs/main}

\end{document}
