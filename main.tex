\documentclass[shuuron]{kuee} % ←ここが修士化の本丸

% ===== packages =====
\usepackage{booktabs}
\usepackage{amsmath, amssymb, amsfonts}
\usepackage{mathtools}
\usepackage{algorithm}
\usepackage{algorithmic}
\usepackage[dvipdfmx]{graphicx}
\usepackage{url}
\usepackage{multirow}

% 図の探索パス(必要なら)
\graphicspath{{figures/}}

\DeclareMathOperator*{\argmax}{arg\,max}

% ===== cover info =====
\title{不確実環境における適応的モデルベース強化学習法}
\etitle{Adaptive Model-Based Reinforcement Learning in Uncertain Environments}
\author{志賀 泰斗}
\eauthor{Taito Shiga}
\professor{石井 信 教授}
% course/department は cls 側で修士用が既定値に切り替わりますが、
% 表示を上書きしたいなら \course{...} \department{...} を書けます。:contentReference[oaicite:3]{index=3}
\date{令和8年1月6日}

\begin{document}
\maketitle

\begin{eabstract}
% English abstract here
\end{eabstract}

\tableofcontents

% ===== chapters =====
\chapter{序論}

近年,ロボット技術は産業,物流,災害対応など,
さまざまな分野において実社会への展開が進んでいる。
これらの応用では,
事前にすべての状況を想定して制御則を設計することが困難であり,
環境や状況に応じて柔軟に振る舞いを変化させる能力が
ロボットに求められている。
このような背景から,
試行錯誤を通じて行動方策を獲得する学習的アプローチが
重要性を増している。

その代表的な枠組みとして,
環境との相互作用を通じて意思決定方策を学習する
強化学習(Reinforcement Learning; RL)が注目されてきた。
近年では,深層ニューラルネットワークを関数近似器として用いることで,
高次元かつ非線形な制御問題に対しても
高い性能を示すことが報告されている
\cite{hafner2020mastering,hafner2019dream,lee2020stochastic}。
RLは,事前に環境モデルを厳密に設計することなく,
複雑な制御問題に適用可能である点から,
実世界ロボットへの応用が期待されている。

一方で,実社会での運用を想定した場合,
実環境における試行錯誤には
安全性やコストの観点から大きな制約が存在する。
このため,
限られたデータから効率的に学習可能な手法が求められる。
この文脈において,
環境の将来挙動を内部モデルとして学習し,
その予測に基づいて行動を計画する
モデルベース強化学習
(Model-Based Reinforcement Learning; MBRL)
は,サンプル効率の観点から
実運用に適した手法として期待されてきた
\cite{NIPS2014_c7c9344b}。
近年では,
高次元観測を潜在空間に写像し,
その空間上で将来の行動系列を計画する手法が提案されており,
TD-MPC2はその代表的な手法の一つである。

しかし,実環境への適用を考えた場合,
既存の多くのMBRL手法は依然として重要な課題に直面している。
モデルベース強化学習は,
学習時に獲得された環境ダイナミクスに基づいて計画を行うため,
物理パラメータが変動した場合,
その影響が計画性能に直接的に現れ, 性能劣化を引き起こすことがある。

この問題が顕在化する代表的な例の一つが,
シミュレーション環境で学習した方策を
実環境へと転移する際に生じる
Sim2Real Gap である。
シミュレーションと実環境の間には,
物理パラメータや接触特性などに差異が存在するため,
転移後に次状態予測の誤差が増大し,
計画の破綻や制御性能の低下を招くことがある。

さらに,
学習と実行の環境が一致している場合であっても,
実環境内では時間の経過とともに
ダイナミクスが変化する可能性がある。
ロボットの摩耗や経年劣化,
個体差,部品交換などにより,
運用中に物理パラメータが変動し,
学習時とは異なるダイナミクスに直面することがある。
このような実環境内での変化もまた,
MBRL手法の性能劣化を引き起こす要因となる。

このように,
Sim2Real における転移時のギャップと,
実環境内で生じるダイナミクス変動は,
発生要因や時間的スケールは異なるものの,
いずれも物理パラメータの変動によって
学習時の世界モデルと実行時の環境ダイナミクスが乖離する点で共通している。
固定的な世界モデルに基づく計画は,
この乖離に対して脆弱であり,
結果として性能の劣化を招きやすい。

以上の背景から,
実運用上起こりうる物理パラメータの変動に対して,
性能劣化を抑制可能なMBRL手法の構築が重要である。
本研究では,
TD-MPC2を基盤とし,
過去の挙動履歴から環境に固有な物理特性を推定し,
その情報を世界モデルの予測に反映させることで,
環境ダイナミクスの変化に適応する
モデルベース強化学習の枠組みを提案する。
本手法では,
環境特性の同定と,
それに基づくダイナミクス予測とを
概念的に分離して扱うことで,
物理パラメータ変動下においても
計画性能を維持することを目指す。

本研究の主な貢献を以下に示す。

\begin{itemize}
  \item \textbf{物理パラメータ変動下における課題構造の整理}  
  
  物理パラメータの変動が,
  モデルベース強化学習の計画性能に与える影響を整理し,
  固定的な世界モデルに基づく手法の限界を明確にした。

  \item \textbf{環境特性同定に基づく適応的MBRL手法の提案}  
  
  TD-MPC2を基盤とし,
  環境に固有な物理特性を明示的に同定し,
  その情報を世界モデルに反映させる
  適応的なモデルベース強化学習手法を提案した。

  \item \textbf{物理パラメータ変動に対する有効性の検証} 
 
  Pendulum-SwingupおよびWalker-Walkなどの制御タスクを用いて,
  物理パラメータ変動下における性能を評価し,
  提案手法が非適応型ベースラインに対して
  有効であることを示した。
\end{itemize}


\begin{acknowledgements}
% 謝辞
本研究を進めるに際して, 指導教員の石井信教授, 国
際電気通信基礎技術研究所 (ATR) の久保顕大さんから多くの助言, 指導を頂きま
した. 厚く感謝を申し上げます. また, 研究の準備および議論を通じて協力して頂
いた石井研究室の皆様にも感謝申し上げます.
\end{acknowledgements}

% \appendix
\chapter{Experimental Details}

\section{Hyperparameter Settings}
% tables/hyperparameters_tdmpc2.tex
% Requires: \usepackage{booktabs}

\begin{table}[htbp]
  \centering
  \caption{Hyperparameter settings. We directly apply settings in~\cite{hansen2023td} for the shared hyperparameters without further tuning. We share the same setting across all tasks demonstrated before.}
  \label{tab:hyperparameters}
  \footnotesize
  \setlength{\tabcolsep}{6pt}
  \renewcommand{\arraystretch}{1.15}

  \begin{tabular}{l|c}
    \toprule
    \textbf{Hyperparameter} & \textbf{Value} \\
    \midrule

    \multicolumn{2}{l}{\textbf{Training}} \\
    \midrule
    Learning rate & $3 \times 10^{-4}$ \\
    Batch size & 256 \\
    Buffer size & 1,000,000 \\
    Sampling & Uniform \\
    Reward loss coefficient ($c_r$) & 0.1 \\
    Value loss coefficient ($c_q$) & 0.1 \\
    Consistency loss coefficient ($c_d$) & 20 \\
    Discount factor ($\gamma$) & 0.99 \\
    Target network update rate & 0.5 \\
    Gradient Clipping Norm & 20 \\
    Optimizer & Adam \\
    \midrule

    \multicolumn{2}{l}{\textbf{Planner}} \\
    \midrule
    MPPI Iterations & 6 \\
    Number of samples & 512 \\
    Number of elites & 64 \\
    Number policy rollouts & 24 \\
    horizon & 3 \\
    Minimum planner std & 0.05 \\
    Maximum planner std & 2 \\
    \midrule

    \multicolumn{2}{l}{\textbf{Actor}} \\
    \midrule
    Minimum policy log std & $-10$ \\
    Maximum policy log std & 2 \\
    Entropy coefficient ($\alpha$) & $1 \times 10^{-4}$ \\
    Prior constraint coefficient ($\beta$) & 1.0 \\
    Scale Threshold ($s$) & 2.0 \\
    \midrule

    \multicolumn{2}{l}{\textbf{Critic}} \\
    \midrule
    Q functions Ensemble & 5 \\
    Number of bins & 101 \\
    Minimum value & $-10$ \\
    Maximum value & 10 \\
    \midrule

    \multicolumn{2}{l}{\textbf{Architecture (5M)}} \\
    \midrule
    Encoder layers & 2 \\
    Encoder dimension & 256 \\
    MLP hidden layer dimension & 512 \\
    Latent space dimension & 512 \\
    Task embedding dimension & 96 \\
    Q function drop out rate & 0.01 \\
    MLP activation & Mish \\
    MLP Normalization & LayerNorm \\
    \bottomrule
  \end{tabular}
\end{table}

\clearpage

\section{Task Specifications}
\label{sec:appendix_task_specs}

本節では,第\ref{chap:experiment}章で用いた評価タスクの詳細仕様について述べる.

\subsection{Pendulum-Swingup タスク仕様}
\label{subsec:appendix_pendulum}

\subsubsection{環境パラメータ}

\begin{table}[h]
\centering
\caption{Pendulum-Swingup 環境パラメータ}
\label{tab:appendix_pendulum_params}
\begin{tabular}{lc}
\toprule
\textbf{パラメータ} & \textbf{値} \\
\midrule
制御周期 & 0.02 s (50 Hz) \\
Action repeat & 2 \\
エピソード長 & 500 steps \\
観測次元 & 3 \\
行動次元 & 1 \\
\midrule
\multicolumn{2}{l}{\textit{物理パラメータ(デフォルト)}} \\
振子質量 & 1.0 kg \\
関節damping & 0.001 \\
重力加速度 & $-9.81$ m/s$^2$ \\
タイムステップ & 0.01 s \\
\bottomrule
\end{tabular}
\end{table}

\subsubsection{観測空間}

観測 $\mathbf{o}_t \in \mathbb{R}^3$ は以下から構成される:
\begin{itemize}
\item \textbf{orientation} (2次元): 振子角度 $\theta$ の $(\cos\theta, \sin\theta)$ 表現
\item \textbf{velocity} (1次元): 角速度 $\dot{\theta}$ [rad/s]
\end{itemize}

\subsubsection{行動空間}

行動 $\mathbf{a}_t \in [-1, 1]$ は振子軸に加えるトルクを表す1次元連続値である.

\subsubsection{報酬関数}

時刻 $t$ における報酬 $r_t$ は,上向き位置からの角度距離 $d_t$ に基づき,
\begin{align}
r_t = \mathrm{tolerance}(d_t; \text{bounds}=(0, 0), \text{margin}=\pi, \text{sigmoid}=\text{`cosine'}),
\end{align}
と定義される.ここで,$\mathrm{tolerance}(\cdot)$ は DMControl における滑らかな許容関数であり,
上向き位置 ($d_t=0$) で $r_t=1$,下向き位置 ($d_t \approx \pi$) で $r_t \approx 0$ となる.
報酬範囲は $[0, 1]$ である.

\subsubsection{ランダム化設定}

Domain Randomization 訓練では,エピソードごとに振子質量を一様分布 $\mathcal{U}(0.5, 2.5)$ kg からサンプリングする.


\subsection{Walker-Walk タスク仕様}
\label{subsec:appendix_walker}

\subsubsection{環境パラメータ}

\begin{table}[h]
\centering
\caption{Walker-Walk 環境パラメータ}
\label{tab:appendix_walker_params}
\begin{tabular}{lc}
\toprule
\textbf{パラメータ} & \textbf{値} \\
\midrule
制御周期 & 0.025 s (40 Hz) \\
Action repeat & 2 \\
エピソード長 & 500 steps (25 s 相当) \\
観測次元 & 24 \\
行動次元 & 6 \\
\midrule
\multicolumn{2}{l}{\textit{物理パラメータ(デフォルト)}} \\
Torso 質量 & 3.34 kg \\
床摩擦係数 & [0.7, 0.1, 0.1] \\
関節damping & 0.1 \\
関節armature & 0.01 \\
重力加速度 & $-9.81$ m/s$^2$ \\
タイムステップ & 0.0025 s \\
\bottomrule
\end{tabular}
\end{table}

\subsubsection{観測空間}

観測 $\mathbf{o}_t \in \mathbb{R}^{24}$ は以下から構成される:
\begin{itemize}
\item \textbf{orientations} (14次元): 胴体および各関節の姿勢($\sin$, $\cos$ 表現)
\item \textbf{height} (1次元): 胴体の地面からの高さ
\item \textbf{velocity} (9次元): 胴体の線速度・角速度,および各関節の角速度
\end{itemize}

\subsubsection{行動空間}

行動 $\mathbf{a}_t \in [-1, 1]^6$ は,左右の股関節・膝関節・足首関節(hip, knee, ankle)に加えるトルクを表す6次元連続値である.

\begin{table}[h]
\centering
\caption{Walker-Walk の行動空間}
\label{tab:appendix_walker_action}
\begin{tabular}{clcc}
\toprule
\textbf{Index} & \textbf{関節名} & \textbf{デフォルトgear} & \textbf{範囲} \\
\midrule
0 & right\_hip & 100 & $[-1, 1]$ \\
1 & right\_knee & 50 & $[-1, 1]$ \\
2 & right\_ankle & 20 & $[-1, 1]$ \\
3 & left\_hip & 100 & $[-1, 1]$ \\
4 & left\_knee & 50 & $[-1, 1]$ \\
5 & left\_ankle & 20 & $[-1, 1]$ \\
\bottomrule
\end{tabular}
\end{table}

実際に環境に加わるトルクは,行動値 $a_i$ と対応する gear 値 $g_i$ の積 $a_i \times g_i$ として計算される.

\subsubsection{報酬関数}

時刻 $t$ における報酬 $r_t$ は,立位報酬 $r_{\mathrm{stand}}$ と移動報酬 $r_{\mathrm{move}}$ の組み合わせとして,
\begin{align}
r_t = \frac{r_{\mathrm{stand}} \cdot (5 r_{\mathrm{move}} + 1)}{6},
\end{align}
と定義される.

立位報酬は,胴体高さ $h_t$ と上向き度 $u_t \in [-1,1]$ に基づき,
\begin{align}
r_{\mathrm{stand}} &= \frac{3 r_{\mathrm{height}} + r_{\mathrm{upright}}}{4}, \\
r_{\mathrm{height}} &= \mathrm{tolerance}(h_t; \text{bounds}=(1.2, \infty), \text{margin}=0.6), \\
r_{\mathrm{upright}} &= \frac{1 + u_t}{2},
\end{align}
とする.

移動報酬は,水平方向速度 $v_t$ に基づき,
\begin{align}
r_{\mathrm{move}} = \mathrm{tolerance}(v_t; \text{bounds}=(1.0, \infty), \text{margin}=0.5, \text{sigmoid}=\text{`linear'}),
\end{align}
と定義する.報酬範囲は理論的には $[0, 1]$ である.

\subsubsection{ランダム化設定}

本研究では,Walker-Walk において2種類の物理パラメータ摂動を用いた:

\paragraph{Torso 質量のランダム化}
エピソードごとに Torso 質量を一様分布 $\mathcal{U}(0.5 \times 3.34, 2.5 \times 3.34)$ kg = $\mathcal{U}(1.67, 8.35)$ kg からサンプリングする.これは荷物運搬やペイロード変化を模擬する.

\paragraph{Actuator gear のランダム化}
エピソードごとに全アクチュエータの gear 値に対して,共通のスケール $s \sim \mathcal{U}(0.4, 1.4)$ を乗じる.これはモーター劣化やバッテリー消耗を模擬する.

\subsubsection{動的環境設定}

動的環境評価では,Walker-Walk において actuator gear スケールがエピソード内で線形に減衰する設定を用いた.初期値 $s_0=1.0$ から最終値 $s_{T}=0.4$ まで,500ステップにわたり線形減衰させる:
\begin{align}
s_t = 1.0 + (0.4 - 1.0) \times \frac{t}{T-1},
\end{align}
ただし $T=500$ である.この設定は,バッテリーの連続消耗などを模擬する.


\clearpage
\section{Implementation Details}
\label{sec:appendix_implementation}

本節では,第\ref{chap:proposal}章で提案した手法の実装上の詳細について述べる.

\subsection{システムアーキテクチャ}
\label{subsec:appendix_system_architecture}

提案手法は,TD-MPC2 に明示的物理パラメータ推定モジュールを追加し,推定結果を条件として計画および制御に反映する構成をとる.システム全体は以下の2つのフェーズから構成される:

\paragraph{フェーズ1: 物理パラメータ推定}
過去 $K$ ステップ分の観測–行動履歴 $(\mathbf{o}_{t-K:t}, \mathbf{a}_{t-K:t-1})$ を GRU エンコーダに入力し,物理パラメータの潜在表現 $\hat{\mathbf{c}}_{\mathrm{phys}}$ を推定する.

\paragraph{フェーズ2: 条件付き計画}
推定された $\hat{\mathbf{c}}_{\mathrm{phys}}$ を世界モデル(dynamics, reward, Q-functions, policy prior)の条件として与え,MPPI プランナーにより行動系列を最適化する.

\paragraph{勾配の分離}
推定器とプランナーの学習を独立に保つため,推定された $\hat{\mathbf{c}}_{\mathrm{phys}}$ をプランナーに渡す際に勾配を切断する(\texttt{.detach()}).これにより,プランナーの学習が推定器を不安定化することを防ぐ.


\subsection{物理パラメータ推定器}
\label{subsec:appendix_physics_estimator}

\subsubsection{GRU アーキテクチャ}

物理パラメータ推定器は,観測–行動履歴から物理パラメータを回帰する GRU ベースのニューラルネットワークである.

\paragraph{入力}
時系列 $\{(\mathbf{o}_{\tau}, \mathbf{a}_{\tau})\}_{\tau=t-K}^{t}$,ただし $K=50$ をコンテキスト長とする.

\paragraph{アーキテクチャ}
\begin{enumerate}
\item \textbf{Input Projection}: 観測と行動を連結し,隠れ次元 256 に線形変換
\begin{align}
\mathbf{h}_{\tau}^{(0)} = \mathrm{ReLU}(\mathrm{LayerNorm}(\mathbf{W}_{\mathrm{in}} [\mathbf{o}_{\tau}; \mathbf{a}_{\tau}]))
\end{align}

\item \textbf{GRU}: 2層,隠れ次元 256,dropout 0.1
\begin{align}
\mathbf{h}_{\tau}^{(l)} = \mathrm{GRU}^{(l)}(\mathbf{h}_{\tau}^{(l-1)}), \quad l=1,2
\end{align}

\item \textbf{Output Head}: 最終ステップの隠れ状態から物理パラメータを回帰
\begin{align}
\hat{\mathbf{c}}_{\mathrm{phys}} = \tanh(\mathbf{W}_{\mathrm{out2}} \, \mathrm{ReLU}(\mathrm{LayerNorm}(\mathbf{W}_{\mathrm{out1}} \mathbf{h}_{t}^{(2)})))
\end{align}
\end{enumerate}

出力は $\hat{\mathbf{c}}_{\mathrm{phys}} \in [-1, 1]^{d_c}$ の範囲に正規化される.本研究では $d_c=8$ を用いた.

\paragraph{損失関数}
推定器は,真の物理パラメータ $\mathbf{c}_{\mathrm{phys}}^{\mathrm{true}}$ との平均二乗誤差により学習する:
\begin{align}
\mathcal{L}_{\mathrm{aux}} = \|\hat{\mathbf{c}}_{\mathrm{phys}} - \mathbf{c}_{\mathrm{phys}}^{\mathrm{true}}\|_2^2
\end{align}

推定器は独立した Optimizer(Adam, 学習率 $3 \times 10^{-4}$)により更新される.


\subsubsection{物理パラメータの正規化}

環境ラッパー(\texttt{PhysicsParamWrapper})が真の物理パラメータを取得し,以下のように正規化する:
\begin{align}
\mathbf{c}_{\mathrm{phys}}^{\mathrm{normalized}} = \frac{\mathbf{c}_{\mathrm{phys}}^{\mathrm{raw}} - \mathbf{c}_{\mathrm{default}}}{\sigma} \in [-1, 1]
\end{align}
ここで,$\mathbf{c}_{\mathrm{default}}$ はデフォルト値,$\sigma$ はランダム化範囲に基づくスケールである.


\subsection{条件付き世界モデル}
\label{subsec:appendix_conditional_world_model}

提案手法の世界モデルは,推定された物理パラメータ $\hat{\mathbf{c}}_{\mathrm{phys}}$ を条件として受け取る.ベースラインの TD-MPC2 と比較して,各モジュールの入力に $\hat{\mathbf{c}}_{\mathrm{phys}}$ を追加する.

\subsubsection{条件付けの実装}

各ネットワークの入力に物理パラメータを連結する:

\paragraph{Dynamics Model}
\begin{align}
\mathbf{z}_{t+1} = f_{\mathrm{dyn}}([\mathbf{z}_t; \mathbf{a}_t; \mathbf{e}_{\mathrm{task}}; \hat{\mathbf{c}}_{\mathrm{phys}}])
\end{align}

\paragraph{Reward Model}
\begin{align}
\hat{r}_t = f_{\mathrm{rew}}([\mathbf{z}_t; \mathbf{a}_t; \mathbf{e}_{\mathrm{task}}; \hat{\mathbf{c}}_{\mathrm{phys}}])
\end{align}

\paragraph{Q-functions}
\begin{align}
Q^{(i)}(\mathbf{z}_t, \mathbf{a}_t) = f_Q^{(i)}([\mathbf{z}_t; \mathbf{a}_t; \mathbf{e}_{\mathrm{task}}; \hat{\mathbf{c}}_{\mathrm{phys}}]), \quad i=1,\ldots,5
\end{align}

\paragraph{Policy Prior}
\begin{align}
\pi(\mathbf{a}_t | \mathbf{z}_t) = f_{\pi}([\mathbf{z}_t; \mathbf{e}_{\mathrm{task}}; \hat{\mathbf{c}}_{\mathrm{phys}}])
\end{align}

ここで,$[\cdot;\cdot]$ はベクトルの連結を表す.$\mathbf{e}_{\mathrm{task}}$ はタスク埋め込み(マルチタスク設定の場合)である.


\subsection{学習アルゴリズム}
\label{subsec:appendix_training_algorithm}

提案手法の学習は,推定器の学習(教師あり)とプランナーの学習(強化学習)の2フェーズから構成される.

\subsubsection{全体の学習手順}

各ステップにおいて,以下の手順で学習を行う:

\begin{enumerate}
\item \textbf{経験の収集}: 環境とインタラクトし,観測 $\mathbf{o}_t$,行動 $\mathbf{a}_t$,報酬 $r_t$,および真の物理パラメータ $\mathbf{c}_{\mathrm{phys}}^{\mathrm{true}}$ を記録

\item \textbf{Buffer への保存}: 過去 $K$ ステップの履歴とともに経験を保存

\item \textbf{バッチサンプリング}: Buffer から $(H+1)$ ステップのトラジェクトリをサンプル

\item \textbf{推定器の更新}: 
\begin{align}
\theta_{\mathrm{GRU}} \leftarrow \theta_{\mathrm{GRU}} - \alpha_{\mathrm{GRU}} \nabla_{\theta_{\mathrm{GRU}}} \mathcal{L}_{\mathrm{aux}}
\end{align}

\item \textbf{物理パラメータの推定}: 
\begin{align}
\hat{\mathbf{c}}_{\mathrm{phys}} = f_{\mathrm{GRU}}(\{\mathbf{o}_{\tau}, \mathbf{a}_{\tau}\}_{\tau=t-K}^{t}; \theta_{\mathrm{GRU}})
\end{align}

\item \textbf{勾配の切断}: 
\begin{align}
\hat{\mathbf{c}}_{\mathrm{phys}} \leftarrow \mathrm{detach}(\hat{\mathbf{c}}_{\mathrm{phys}})
\end{align}

\item \textbf{プランナーの更新}: $\hat{\mathbf{c}}_{\mathrm{phys}}$ を条件として,TD-MPC2 の損失により世界モデルを更新
\begin{align}
\mathcal{L}_{\mathrm{TDMPC2}} &= \mathcal{L}_{\mathrm{consistency}} + \mathcal{L}_{\mathrm{reward}} + \mathcal{L}_{\mathrm{value}} + \mathcal{L}_{\mathrm{termination}} \\
\theta_{\mathrm{planner}} &\leftarrow \theta_{\mathrm{planner}} - \alpha_{\mathrm{planner}} \nabla_{\theta_{\mathrm{planner}}} \mathcal{L}_{\mathrm{TDMPC2}}
\end{align}
\end{enumerate}

重要な点として,ステップ6の勾配切断により,プランナーの勾配が推定器に逆伝播しない.これにより,各モジュールが独立に学習可能となる.


\subsubsection{Optimizer の設定}

提案手法では3つの独立した Optimizer を用いる:

\begin{table}[h]
\centering
\caption{Optimizer の設定}
\label{tab:appendix_optimizers}
\begin{tabular}{lcc}
\toprule
\textbf{Optimizer} & \textbf{対象パラメータ} & \textbf{学習率} \\
\midrule
GRU Optimizer & 推定器 & $3 \times 10^{-4}$ \\
Planner Optimizer & Encoder, Dynamics, Reward, Q, Termination & $1 \times 10^{-3}$ \\
Policy Optimizer & Policy Prior & $1 \times 10^{-3}$ \\
\bottomrule
\end{tabular}
\end{table}


\subsubsection{教師ラベルの取得}

シミュレーション環境では,真の物理パラメータは環境ラッパーから直接取得可能である.各エピソードの開始時に物理パラメータがサンプリングされ,エピソード中は固定される.この物理パラメータを正規化したものを教師ラベルとして用いる.

実世界への展開においては,キャリブレーションや設計値を教師ラベルとして利用することが考えられる.また,オンライン同定手法(例:最小二乗法)により推定した値を擬似ラベルとして用いることも可能である.

% ===== bibliography =====
% 1) bst を refs/ に置く場合
% \bibliographystyle{refs/kueethesis}
% % 2) bib を refs/ に置く場合(例:refs/main.bib)
% \bibliography{refs/main}

\end{document}
