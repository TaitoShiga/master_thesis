\chapter{提案手法}
\label{chap:proposal}

\section{不確実環境における制御問題の定式化}
\label{sec:problem_reformulation}

本節では,
第\ref{chap:preliminary}章で整理した TD-MPC2 の前提を踏まえ,
不確実な物理環境において生じる制御上の課題を明確化する.
その上で,
本研究が採用する基本的な設計方針について述べる.


\subsection{想定する環境不確実性と課題設定}
\label{subsec:uncertainty_setting}

本研究では,
ロボットの質量,関節特性といった
物理パラメータが未知な環境を環境不確実性として定義する.

このような物理パラメータは,
単一時刻の観測から直接推定することは困難であり,
エージェントの行動に対する環境の応答として,
時間的に蓄積される挙動の違いの中に現れる.
したがって,
現在の観測のみに基づく意思決定では,
環境の特性を十分に捉えることができない.

一方で,
エージェントが過去の観測および行動の履歴を考慮することで,
環境固有の物理特性に関する情報を間接的に取得できる可能性がある.
この点は,
部分観測環境における制御問題として捉えることができる.

\paragraph{物理パラメータの時間変化に関する想定}

本研究では,
物理パラメータの時間変化のパターンとして,
以下の二つの状況を想定する.

\begin{enumerate}
\item \textbf{エピソード間の変動}:
物理パラメータはエピソード開始時に決定され,
単一エピソード内では一定に保たれるが,
エピソードごとに異なる値をとる場合.
これは,
異なる個体や環境条件下での制御を想定したものであり,
学習したモデルが, 未知だが固定された物理パラメータ環境に対してどの程度汎化可能かを評価する設定として用いる.

\item \textbf{エピソード内の変動}:
物理パラメータがエピソード実行中に時間的に変化する場合.
これは,
バッテリー消耗によるアクチュエータ出力の低下や,
摩耗による機構特性の変化など,
実環境において生じうる動的な変動を想定したものである.
このような状況では,
推定器がオンラインで変化を追従する必要がある.
\end{enumerate}

提案手法の設計は,
いずれのパターンにも対応可能な構成を採用しており,
第\ref{chap:experiment}章の実験では両方の状況における性能を検証する.


\subsection{既存 TD-MPC2 における暗黙的仮定とその限界}
\label{subsec:tdmpc2_limitation}

第\ref{chap:preliminary}章で述べたように,
TD-MPC2 は,
学習された世界モデル上での将来予測と,
モデル予測制御に基づく計画を統合した強力な制御手法である.
しかし,
TD-MPC2 では,
世界モデルが固定された環境ダイナミクスを表現していることが暗黙的に仮定されている.

この仮定の下では,
学習時に想定していない物理パラメータの変動が生じた場合,
世界モデルによる将来予測と実際の環境挙動との間に乖離が生じる.
この乖離は,
モデル予測制御におけるロールアウトを通じて累積し,
計画の信頼性を低下させる要因となる.

\subsection{本研究の基本方針:不確実性の明示的同定}
\label{subsec:design_policy}

以上の背景を踏まえ,
本研究では,
環境の不確実性を単に吸収すべきノイズとして扱うのではなく,
\textbf{明示的に同定すべき対象}として捉える立場を取る.
すなわち,
環境の物理パラメータを推定し,
その推定結果を制御および計画に反映させることで,
将来予測の整合性を向上させることを目的とする.

本研究の設計目標は,
単に平均的な性能を向上させることではなく,
不確実環境下における
(i) 計画の信頼性,
(ii) 推定結果の解釈性
を同時に向上させることである.
この方針に基づき,
次節以降では,
明示的な物理パラメータ推定器と,
それを用いた適応的なモデル予測制御の構成について述べる.


\section{提案手法のシステム構成}
\label{sec:system_overview}

本節では,
本研究で提案する明示的物理パラメータ同定に基づく制御手法の
システム全体構成について述べる.
提案手法は,
既存の TD-MPC2 を制御器の中核として保持しつつ,
相互作用履歴から環境の物理パラメータを推定するモジュールを追加し,
その推定結果を計画過程に条件として与える構成を採用する.


\subsection{全体アーキテクチャ}
\label{subsec:overall_architecture}

提案手法の全体アーキテクチャを図\ref{fig:architecture}に示す.
本手法は,
大きく分けて
(i) 物理パラメータ推定モジュールと,
(ii) 条件付きモデル予測制御モジュール
の二つから構成される.

\begin{figure}[tb]
\centering
\includegraphics[width=0.9\linewidth]{figures/architecture.png}
\caption[提案手法の全体アーキテクチャ]{%
\centering\textbf{提案手法の全体アーキテクチャ}\\[1ex]
\raggedright\small
環境から得られる観測 $\mathbf{o}_t$ は,
まず明示的システム同定モジュールに入力され,
過去 $K$ ステップ分の観測–行動履歴
$(\mathbf{o}_{t-K:t}, \mathbf{a}_{t-K:t-1})$
を用いて GRU エンコーダにより
物理パラメータの潜在表現
$\hat{\mathbf{c}}_{\mathrm{phys}}$ が推定される.
同時に,現在の観測 $\mathbf{o}_t$ は
TD-MPC2 コントローラに入力され,
エンコーダ $f_{\mathrm{enc}}$ により潜在状態
$\mathbf{s}_t$ に写像される.
推定された物理パラメータ
$\hat{\mathbf{c}}_{\mathrm{phys}}$ は,
潜在ダイナミクスモデル,
報酬・価値モデル,
および方策事前分布に条件として与えられ,
物理パラメータ推定結果を反映した潜在ロールアウトが実行される.
MPPI プランナは,
条件付き世界モデルに基づいて行動系列を評価・最適化し,
得られた行動 $\mathbf{a}_t$ を環境に適用する.
}
\label{fig:architecture}
\end{figure}

制御器の中核には,
第\ref{chap:preliminary}章で述べた TD-MPC2 をそのまま用いる.
すなわち,
観測 \(\mathbf{o}_t\) はエンコーダによって潜在状態 \(\mathbf{s}_t\) に写像され,
潜在空間上の世界モデルを用いたロールアウトと
モデル予測制御により行動が決定される.

提案手法では,
制御とは独立に,
過去の観測および行動の履歴を入力として
環境の物理パラメータを推定する推定モジュールを追加する.
この推定結果は,
制御器内部の潜在状態とは別の変数として保持され,
世界モデルおよび計画過程に条件として与えられる.詳細は\ref{subsec:conditional_dynamics}節で述べる.

この構成により,
TD-MPC2 が本来持つ
高い計画性能および学習安定性を損なうことなく,
環境の物理的差異を明示的に反映した制御が可能となる.


\subsection{推定モジュールと制御モジュールの結合設計}
\label{subsec:module_coupling}

提案手法では,
物理パラメータ推定モジュールと制御モジュールを独立させた疎結合な構成を採用した.
推定された物理パラメータ \(\hat{\mathbf{c}}_{\mathrm{phys}}\) は,
制御器内部の潜在状態表現には統合せず,
各予測モジュールに対する「条件付け変数(conditioning variable)」として扱われる.
具体的には,
エンコーダを除く世界モデルの構成要素(Dynamics, Reward, Value)および行動事前分布(Policy Prior)の入力として \(\hat{\mathbf{c}}_{\mathrm{phys}}\) を明示的に注入する設計としている.

この設計の核心は,
推定プロセスを制御器の外部に配置することで,
物理パラメータ推定モジュールの学習が制御器側の表現獲得や価値学習を不安定化させることを防ぐ点にある.
学習時には推定値 \(\hat{\mathbf{c}}_{\mathrm{phys}}\) を介した勾配の逆伝播を遮断(detach)する構造をとることで,
既存の TD-MPC2 が備える優れた学習特性を損なうことなく,
環境特性に応じた適応能力のみを付加することが可能となる.
これにより,
エージェントは同一の潜在状態および行動に対しても,
同定された物理パラメータに応じて将来予測や計画結果を動的に変化させることができる.

このように推定と制御の役割を概念的に分離した構成は,
(i) 物理パラメータ同定の明示的な解釈性,
(ii) 不確実環境下における適応プロセスの安定性,
(iii) システム全体のモジュール性と実装の単純性
を同時に担保するものである.


\section{明示的物理パラメータ推定器}
\label{sec:explicit_sysid}

本節では,
相互作用履歴から環境の物理パラメータを推定するために導入した,
明示的物理パラメータ推定器について述べる.


\subsection{相互作用履歴に基づく推定問題の定式化}
\label{subsec:sysid_formulation}

時刻 \(t\) における過去 \(K\) ステップ分の観測および行動の履歴を,
次のように定義する.

\begin{align}
\mathcal{H}_t
=
\{(\mathbf{o}_{t-K}, \mathbf{a}_{t-K}), \ldots, (\mathbf{o}_{t-1}, \mathbf{a}_{t-1})\}.
\end{align}

本研究では,
この履歴情報 \(\mathcal{H}_t\) に基づいて,
環境に固有な物理パラメータ
\(\mathbf{c}_{\mathrm{phys}} \in \mathbb{R}^d\)
を推定する問題を考える.
ここで,
\(\mathbf{c}_{\mathrm{phys}}\) は,
質量や関節特性など,
制御性能に大きな影響を与えるが,
単一時刻の観測からは直接得られない量を表す.

そして推定問題は次の写像として表現できる.

\begin{align}
\hat{\mathbf{c}}_{\mathrm{phys},t}
=
g_{\theta}(\mathcal{H}_t),
\end{align}

ここで,
\(g_{\theta}(\cdot)\) は
履歴から物理パラメータを出力する推定器を表す.


\subsection{GRU による履歴情報のエンコーディング}
\label{subsec:gru_encoding}

相互作用履歴 \(\mathcal{H}_t\) は時系列データであり,
物理パラメータに関する情報は,
行動に対する環境応答の違いとして時間的に現れる.
このような時系列依存性を捉えるため,
本研究では再帰型ニューラルネットワーク(RNN)の一種である
Gated Recurrent Unit (GRU) を用いて履歴をエンコードする.

GRU は,
入力系列を逐次的に処理し,
隠れ状態 \(\mathbf{h}_t\) に過去の情報を集約する.
本研究では,
各時刻の入力を
観測と行動の組 \((\mathbf{o}_{t}, \mathbf{a}_{t})\)
として GRU に与え,
最終的な隠れ状態を
履歴全体の要約表現として用いる.

GRU を採用した理由は,
時系列情報を扱う RNN 系モデルの中でも構造が簡潔であり,
比較的少ないパラメータ数で安定した学習が可能であるためである.



\subsection{物理パラメータ回帰モジュール}
\label{subsec:parameter_regression}

GRU により得られた隠れ状態 \(\mathbf{h}_t\) を入力として,
物理パラメータを回帰するモジュールを構成する.
具体的には,
\(\mathbf{h}_t\) を全結合層に入力し,
推定値 \(\hat{\mathbf{c}}_{\mathrm{phys},t}\) を出力する.

本研究では,
実験設定の単純化のため,
推定対象の物理パラメータを一次元に限定する.
提案手法の構成は推定次元数に依存せず,
多次元の物理パラメータ推定にも同様に適用可能である.



\section{推定に基づく適応的計画}
\label{sec:adaptive_planning}

本節では,
前節で推定した物理パラメータを用いて,
モデル予測制御をどのように適応させるかについて述べる.
提案手法では,
推定結果を世界モデルおよび計画過程に条件として与えることで,
環境変動に応じた将来予測と行動計画を実現する.


% \subsection{物理パラメータの正規化と注入方法}
% \label{subsec:parameter_injection}

% 推定された物理パラメータ
% \(\hat{\mathbf{c}}_{\mathrm{phys},t}\) は,
% 世界モデルおよび計画アルゴリズムに入力される前に,
% 正規化処理を施す.
% これは,
% 学習時に用いたパラメータ分布と
% 推定値のスケールを整合させるためである.

% このような条件付けは,
% 物理パラメータが状態とは異なる時間スケールで変化するという仮定と整合的であり,
% 推定誤差が潜在状態推定に直接影響することを防ぐ効果も持つ.


\subsection{条件付き世界モデルによる将来予測}
\label{subsec:conditional_dynamics}

物理パラメータを条件として導入した場合,
潜在空間における状態遷移は次のように表される.

\begin{align}
\mathbf{s}_{t+1}
=
f(\mathbf{s}_t, \mathbf{a}_t, \hat{\mathbf{c}}_{\mathrm{phys},t}).
\end{align}

この条件付き遷移モデルにより,
同一の潜在状態および行動であっても,
物理パラメータの違いに応じて
異なる将来状態が予測される.
これにより,
世界モデルは
環境ごとのダイナミクス差異を明示的に表現することが可能となる.

条件付き世界モデルを用いたロールアウトでは,
推定された物理パラメータを固定したまま,
複数ステップ先までの潜在状態遷移を予測する.
この予測結果に基づき,
報酬および価値が評価され,
モデル予測制御における行動系列の最適化が行われる.


\subsection{推定パラメータ条件下での軌道計画}
\label{subsec:conditional_mppi}

TD-MPC2 では,
将来の行動系列を確率的にサンプリングし,
各軌道の累積コストに基づいて行動列を更新する
Model Predictive Path Integral(MPPI)制御
\cite{williams2015modelpredictivepathintegral}
が用いられている.
MPPI は,
勾配計算を必要とせず,
非線形・非凸なダイナミクスに対しても適用可能な
サンプリングベースのモデル予測制御手法である.

MPPI における行動更新は,
サンプルされた行動摂動 $\{\delta \mathbf{u}_t^{(k)}\}_{k=1}^K$ に対して,
各軌道のコスト $S^{(k)}$ を用いた指数重み付き平均として与えられる:
\begin{equation}
\mathbf{u}_t \leftarrow \mathbf{u}_t +
\frac{\sum_{k=1}^K \exp\!\left(-\frac{1}{\lambda} S^{(k)}\right)
\delta \mathbf{u}_t^{(k)}}
{\sum_{k=1}^K \exp\!\left(-\frac{1}{\lambda} S^{(k)}\right)} .
\end{equation}
ここで,$S^{(k)}$ は各行動系列に対する
ロールアウト結果に基づく累積コストであり,
将来状態の予測に依存する.

提案手法では,
このロールアウトに用いる世界モデルを,
推定された物理パラメータ $\hat{\boldsymbol{\theta}}_t$ で条件付けることで,
各サンプル軌道の状態遷移およびコスト評価を行う.
すなわち,
\begin{equation}
\mathbf{x}_{t+1} = f(\mathbf{x}_t, \mathbf{u}_t ; \hat{\boldsymbol{\theta}}_t)
\end{equation}
として予測された軌道に基づき,
$S^{(k)}$ を計算する.

この操作は,
MPPI の更新式における「コスト評価の部分」にのみ影響し,
サンプリング・重み付け・平均化という最適化構造そのものは変化しない.
したがって,
推定パラメータを条件として導入しても,
MPPI の理論的枠組みや最適化手順が破綻することはなく,
推定された環境モデルに整合した軌道計画が自然に実現される.

さらに,
MPPI は多数のサンプル軌道を並列に評価するため,
推定パラメータに誤差が含まれる場合でも,
単一のモデル予測に過度に依存することなく,
相対的に低コストな行動系列を選択できる.
この性質により,
提案手法は完全な同定が得られない状況においても,
安定した制御性能を維持することが可能である.


\section{学習アルゴリズム}
\label{sec:learning_algorithm}

本節では,
提案手法における学習アルゴリズムについて述べる.
提案手法は,
物理パラメータ推定器と,
条件付き TD-MPC2 コントローラという
異なる役割を持つ二つのモジュールから構成されている.
本研究では,
それぞれの目的および時間スケールの違いを考慮し,
両者を分離して学習する方針を採用する.


\subsection{物理パラメータ推定器の学習}
\label{subsec:estimator_training}

物理パラメータ推定器は,
相互作用履歴 \(\mathcal{H}_t\) を入力として,
環境に固有な物理パラメータ
\(\mathbf{c}_{\mathrm{phys}}\) を出力する回帰モデルとして構成される.
本研究では,
シミュレーション環境において
真の物理パラメータが既知であることを利用し,
推定器を教師あり学習によって訓練する.

具体的には,
履歴 \(\mathcal{H}_t\) と対応する真の物理パラメータ
\(\mathbf{c}_{\mathrm{phys}}\) の組を学習データとし,
次の二乗誤差損失を最小化する.

\begin{align}
\mathcal{L}_{\mathrm{est}}
=
\left\|
\hat{\mathbf{c}}_{\mathrm{phys},t}
-
\mathbf{c}_{\mathrm{phys}}
\right\|^2 .
\end{align}

ここで,
\(\hat{\mathbf{c}}_{\mathrm{phys},t}\) は
推定器による予測値を表す.
物理パラメータは,
単一エピソード内では一定とし,
履歴の任意の時刻において同一の教師信号を用いた.

\subsection{条件付き制御器の学習}
\label{subsec:controller_training}

条件付き制御器は,
推定された物理パラメータを条件として受け取り,
潜在空間上の世界モデルおよび価値関数を学習する.
学習アルゴリズムの基本構造は,
第\ref{chap:preliminary}章で述べた
標準的な TD-MPC2 と同一である.

具体的には,
観測 \(\mathbf{o}_t\) から得られた潜在状態
\(\mathbf{s}_t\) に対して,
条件付き遷移モデル
\(\mathbf{s}_{t+1} = f(\mathbf{s}_t, \mathbf{a}_t, \hat{\mathbf{c}}_{\mathrm{phys},t})\)
を用いた多段階予測を行い,
報酬モデルおよび価値関数を
TD誤差に基づいて更新する.

\subsection{勾配分離による学習安定化}
\label{subsec:gradient_decoupling}

提案手法では,
物理パラメータ推定器と
条件付き TD-MPC2 コントローラの間で,
勾配を明示的に分離する.
具体的には,
推定器の出力
\(\hat{\mathbf{c}}_{\mathrm{phys},t}\) を
制御器に入力する際に,
勾配を遮断する処理を行う.

この勾配分離の導入により,
制御タスクの達成度に基づく誤差信号が推定モジュールの学習
プロセスへと逆伝搬することを防ぐ.
これにより,学習初期の不安定な制御勾配が物理パラメータ
の推定精度を損なう事態を回避し,推定と制御という二つの学習プロセス
が相互に悪影響を及ぼしあいながら不安定化する事態を未然に防ぐ設計となっている.

提案手法では,
推定器は
「物理パラメータを推定する」という明確な目的のもとで学習され,
制御器は
「推定された条件下で最適な行動を選択する」
という役割に専念する.
この役割分担に基づく学習分離により,
全体として安定した学習および制御性能が得られると考えられる.
