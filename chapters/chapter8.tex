\chapter{結論}
\label{chap:conclusion}

\section{研究の総括}
\label{sec:summary}

本研究では,
物理パラメータが未知な環境において,
モデルベース強化学習における計画性能を維持するための
適応的制御手法を提案した.

既存のモデルベース強化学習手法は,
学習時に想定された物理パラメータから
実行時の環境が乖離した場合,
世界モデルの予測精度が低下し,
計画の信頼性が損なわれるという課題を有していた.

これに対し本研究では,
環境の物理パラメータを過去の相互作用履歴から明示的に推定し,
その推定結果を世界モデルおよび計画過程に条件として与える枠組みを構築した.
推定と制御を概念的に分離する設計により,
学習の安定性を保ちつつ環境変動への適応能力を付加することが可能となった.

Pendulum-Swingup および Walker-Walk タスクを用いた実験により,
提案手法は高次元タスクにおいて Non-adaptive および DRモデル を上回る性能を達成し,
動的環境においても変化に追従する挙動を示すことが確認された.
また,真の物理パラメータを与えた Oracle TD-MPC2 が
環境変動下でも高い性能を維持することを実証し,
物理パラメータの明示的な推定が計画の信頼性向上に寄与することを示した.


\section{今後の課題}
\label{sec:future_work}

本研究の提案手法には以下の課題が残されている.

\paragraph{推定ラグの影響}
エピソード初期における推定値の収束遅延は,
特に短時間タスクや単純なタスクにおいて累積報酬を低下させる.
推定の収束速度向上のため,
初期化戦略や推定器アーキテクチャの改良が必要である.

\paragraph{多次元パラメータへのスケーラビリティ}
本研究では単一パラメータの推定に焦点を当てたが,
実環境では複数の物理パラメータが同時に変化する可能性がある.
多次元パラメータ推定における収束性や,
推定すべきパラメータの優先順位付けの検討が必要である.

\paragraph{実機ロボットへの展開}
シミュレーション環境での検証にとどまっており,
実機展開においては
Sim2Real ギャップ,観測ノイズ,アクチュエータ遅延などへの対処が必要となる.
また,真の物理パラメータへのアクセスが困難な実環境における
推定器の学習戦略の再検討も重要な課題である.

\paragraph{計算コストの削減}
GRU による系列推論は追加の計算コストを伴うため,
リアルタイム制約下での実用性向上のため,
推定器の軽量化や推定頻度の最適化が求められる.

これらの課題に取り組むことで,
実環境における不確実性に対してより頑健かつ適応的なロボット制御システムの実現が期待される.
