\chapter{序論}

近年,ロボット技術は産業,物流,災害対応など,
さまざまな分野において実社会への展開が進んでいる。
これらの応用では,
事前にすべての状況を想定して制御則を設計することが困難であり,
環境や状況に応じて柔軟に振る舞いを変化させる能力が
ロボットに求められている。
このような背景から,
試行錯誤を通じて行動方策を獲得する学習的アプローチが
重要性を増している。

その代表的な枠組みとして,
環境との相互作用を通じて意思決定方策を学習する
強化学習(Reinforcement Learning; RL)が注目されてきた。
近年では,深層ニューラルネットワークを関数近似器として用いることで,
高次元かつ非線形な制御問題に対しても
高い性能を示すことが報告されている
\cite{hafner2020mastering,hafner2019dream,lee2020stochastic}。
RLは,事前に環境モデルを厳密に設計することなく,
複雑な制御問題に適用可能である点から,
実世界ロボットへの応用が期待されている。

一方で,実社会での運用を想定した場合,
実環境における試行錯誤には
安全性やコストの観点から大きな制約が存在する。
このため,
限られたデータから効率的に学習可能な手法が求められる。
この文脈において,
環境の将来挙動を内部モデルとして学習し,
その予測に基づいて行動を計画する
モデルベース強化学習
(Model-Based Reinforcement Learning; MBRL)
は,サンプル効率の観点から
実運用に適した手法として期待されてきた
\cite{NIPS2014_c7c9344b}。
近年では,
高次元観測を潜在空間に写像し,
その空間上で将来の行動系列を計画する手法が提案されており,
TD-MPC2はその代表的な手法の一つである。

しかし,実環境への適用を考えた場合,
既存の多くのMBRL手法は依然として重要な課題に直面している。
モデルベース強化学習は,
学習時に獲得された環境ダイナミクスに基づいて計画を行うため,
物理パラメータが変動した場合,
その影響が計画性能に直接的に現れ, 性能劣化を引き起こすことがある。

この問題が顕在化する代表的な例の一つが,
シミュレーション環境で学習した方策を
実環境へと転移する際に生じる
Sim2Real Gap である。
シミュレーションと実環境の間には,
物理パラメータや接触特性などに差異が存在するため,
転移後に次状態予測の誤差が増大し,
計画の破綻や制御性能の低下を招くことがある。

さらに,
学習と実行の環境が一致している場合であっても,
実環境内では時間の経過とともに
ダイナミクスが変化する可能性がある。
ロボットの摩耗や経年劣化,
個体差,部品交換などにより,
運用中に物理パラメータが変動し,
学習時とは異なるダイナミクスに直面することがある。
このような実環境内での変化もまた,
MBRL手法の性能劣化を引き起こす要因となる。

このように,
Sim2Real における転移時のギャップと,
実環境内で生じるダイナミクス変動は,
発生要因や時間的スケールは異なるものの,
いずれも物理パラメータの変動によって
学習時の世界モデルと実行時の環境ダイナミクスが乖離する点で共通している。
固定的な世界モデルに基づく計画は,
この乖離に対して脆弱であり,
結果として性能の劣化を招きやすい。

以上の背景から,
実運用上起こりうる物理パラメータの変動に対して,
性能劣化を抑制可能なMBRL手法の構築が重要である。
本研究では,
TD-MPC2を基盤とし,
過去の挙動履歴から環境に固有な物理特性を推定し,
その情報を世界モデルの予測に反映させることで,
環境ダイナミクスの変化に適応する
モデルベース強化学習の枠組みを提案する。
本手法では,
環境特性の同定と,
それに基づくダイナミクス予測とを
概念的に分離して扱うことで,
物理パラメータ変動下においても
計画性能を維持することを目指す。

本研究の主な貢献を以下に示す。

\begin{itemize}
  \item \textbf{物理パラメータ変動下における課題構造の整理}  
  
  物理パラメータの変動が,
  モデルベース強化学習の計画性能に与える影響を整理し,
  固定的な世界モデルに基づく手法の限界を明確にした。

  \item \textbf{環境特性同定に基づく適応的MBRL手法の提案}  
  
  TD-MPC2を基盤とし,
  環境に固有な物理特性を明示的に同定し,
  その情報を世界モデルに反映させる
  適応的なモデルベース強化学習手法を提案した。

  \item \textbf{物理パラメータ変動に対する有効性の検証} 
 
  Pendulum-SwingupおよびWalker-Walkなどの制御タスクを用いて,
  物理パラメータ変動下における性能を評価し,
  提案手法が非適応型ベースラインに対して
  有効であることを示した。
\end{itemize}
